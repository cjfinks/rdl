% === TODO ===
% Derive geometric properties of A from Spark Condition
% Motivate "data spread"
% Describe m' > m
% Prove probabilistic statements
% General random sampling - what are the odds of getting k+1 of them in a special support sets?

\documentclass[journal, onecolumn]{IEEEtran}

% *** MATH PACKAGES ***
\usepackage{amsmath, amssymb, amsthm} 
\newtheorem{theorem}{Theorem}
\newtheorem{lemma}{Lemma}
\newtheorem{conjecture}{Conjecture}
\newtheorem{problem}{Problem}
\newtheorem{question}{Question}
\newtheorem{proposition}{Proposition}
\newtheorem{definition}{Definition}
\newtheorem{corollary}{Corollary}
\newtheorem{remark}{Remark}
\newtheorem{example}{Example}


% *** ALIGNMENT PACKAGES ***
\usepackage{array}

% correct bad hyphenation here
\hyphenation{op-tical net-works semi-conduc-tor}

\begin{document}

\title{Combinatorics of Uniqueness in Sparse Dictionary Learning}

\author{Charles~J.~Garfinkle,  Christopher~J.~Hillar%
\thanks{The research of Garfinkle and Hillar was conducted while at the Redwood Center for Theoretical Neuroscience, Berkeley, CA, USA; e-mails: cjg@berkeley.edu, chillar@msri.org.}}%

\maketitle

\begin{abstract}
We study uniqueness in sparse dictionary learning when reconstruction of data is approximate.
\end{abstract}

Fix positive integers $k < m$ and let $\mathcal{S} = \{S_0,\ldots,S_{m-1}\}$, where
\[S_{i} = \{i, i + 1, \ldots, i + (k-1)\}  \ \text{modulo } m, \ \ \ \text{for}  \ \ i = 0, \ldots m -1.\]

\begin{lemma}\label{NonEmptyLemma} Fix positive integers $k < m$ and let $\mathcal{S} = \{S_0,\ldots,S_{m-1}\}$, where for $i = 0, \ldots m -1$,
\[S_{i} = \{i, i + 1, \ldots, i + (k-1)\}  \ \text{mod } m.\]
Suppose there exists a map $\pi: \mathcal{S} \to {\mathbb{Z}/m\mathbb{Z} \choose k}$ such that for all $ \mathcal{I} \in {[m] \choose k}$,
\begin{align}\label{EmptyToEmpty}
 \bigcap_{i \in \mathcal{I}} S_i = \emptyset \Longrightarrow \bigcap_{i \in \mathcal{I}} \pi(S_i) = \emptyset.
\end{align}
%
Then  $\pi(S_i) \cap \cdots \cap \pi(S_{i+(k-1)}) \neq \emptyset$ for all $i \in \mathbb{Z}/m\mathbb{Z}$.
\end{lemma}

\emph{Proof of Lemma \ref{NonEmptyLemma}:} Consider the set $T_m = \{ (i,j) : i \in \mathbb{Z}/m\mathbb{Z}, j \in \pi(S_i) \}$, which has $mk$ elements. By the pigeon-hole principle, there is some $p \in \mathbb{Z}/m\mathbb{Z}$ and at least $k$ distinct $i_1, \ldots, i_k$ such that $\{(i_1, p), \ldots, (i_k, p)\} \subseteq T_m$. Hence, $p \in \pi(S_{i_1}) \cap \cdots \cap \pi(S_{i_k})$ and by \eqref{EmptyToEmpty} there must be some $v \in \mathbb{Z}/m\mathbb{Z}$ such that $v \in S_{i_1} \cap \cdots \cap S_{i_k}$. This is only possible (given $\mathcal{S}$) if $i_1, \ldots, i_k$ are consecutive modulo $\mathbb{Z}/m\mathbb{Z}$, i.e. $\{i_1, \ldots, i_k\} = \{v - (k-1), \ldots, v\}$. 

We now claim there exists no additional $i^* \in \mathbb{Z}/m\mathbb{Z} \setminus \{i_1, \ldots, i_k\}$ such that $p \in \pi(S_{i^*})$. To see why, note that we would then have $p \in \pi(S_{i^*}) \cap \pi(S_{v - (k-1)}) \cap \cdots \cap \pi(S_{v})$ and \eqref{EmptyToEmpty} would imply that every $k$-element subset of $\{i^*\} \cup \{v-(k-1), \ldots, v\}$ is a consecutive set. This is only possible if $m = k+1$; but then there can't have been $k+1$ distinct elements of ${\mathbb{Z}/m\mathbb{Z} \choose k}$ all containing $p$ since there are only ${m-1 \choose m-2}  = m-1 = k$ distinct elements of ${\mathbb{Z}/m\mathbb{Z} \choose m-1}$ which contain $p$. Thus, letting $T_{m-1} \subset T_m$ be the set of elements of $T_m$ not having $p$ as a second coordinate, we have $|T_{m-1}| = (m-1)k$ and the proof follows by iterating these arguments. $\indent \blacksquare$

\begin{lemma}\label{NonEmptyLemma} Let $k \geq 2$ and $m > 2k$ (or $m > 3$ if $k=2$) and suppose there exists a map $\pi: \mathcal{S} \to {\mathbb{Z}/m\mathbb{Z} \choose k}$ such that for all $S, S' \in \mathcal{S}$ we have
\begin{align}\label{PiSltS}
S \cap S' = \emptyset \Longrightarrow \pi(S) \cap \pi(S') = \emptyset .
\end{align}
Then we also have $S \cap S' \neq \emptyset \Longrightarrow \pi(S) \cap \pi(S') \neq \emptyset$ for all $S, S' \in \mathcal{S}$.
\end{lemma}

\emph{Proof of Lemma \ref{NonEmptyLemma}:} Consider the set $T_m = \{ (i,j) : i \in \mathbb{Z}/m\mathbb{Z}, j \in \pi(S_i)) \}$, which has $mk$ elements. By the pigeon-hole principle, there is some $p \in \mathbb{Z}/m\mathbb{Z}$ and $k$ distinct $i_1, \ldots, i_k$ such that $(i_1, p), \ldots, (i_k,p) \in T_m$. Letting $\mathcal{I} = \{i_1, \ldots, i_k\}$, we have $p \in \pi(S_i)$ for all $i \in \mathcal{I}$ and by \eqref{PiSltS} we must have $S_{i} \cap S_{j} \neq \emptyset$ for all $i, j \in \mathcal{I}$. We claim that $\mathcal{I}$ must therefore consist of consecutive integers modulo $m$. To see why, suppose w.l.o.g. that $k-1 \in \mathcal{I}$. Then $\mathcal{I} \subset [0, 2k-2]$. If $k=2$ then we are done; otherwise, suppose that for some $i, j \in \mathcal{I}$, $i > j$ we have $(S_i \cap S_j) \cap \mathcal{I} = \emptyset$. Then $j +(k-1) < i$, i.e. $j \in [0, k-2]$ whereas $i + (k-1) \geq j \geq m > 2k$, i.e. $i \in [k+2, 2k-2]$. Hmm... \textbf{[Proof idea: You can't have three pairwise intersecting sets which don't share a common element.]} Hence there exists some $v \in \mathbb{Z}/m\mathbb{Z}$ such that $\mathcal{I} = \{v - (k-1), \ldots, v\}$ and $p \in \pi(S_{v-(k-1)}) \cap \cdots \cap \pi(S_{v})$. 

Suppose now that there exists some additional $i^* \in \mathbb{Z}/m\mathbb{Z} \setminus \{v-(k-1), \ldots, v\}$ such that $p \in \pi(S_{i^*})$. Then $p \in \pi(S_{i^*}) \cap \pi(S_i)$ for all $i \in \{v-(k-1), \ldots, v\}$. Hence by \eqref{PiSltS} we have $S_{i^*} \cap S_i \neq \emptyset$ for all $i \in \{v-(k-1), \ldots, v\}$ which is impossible since $m \geq 2k$. Thus there can be no such $i^*$. Letting $T_{m-1} \subset T_m$ be the set of elements of $T_m$ not having $p$ as a second coordinate, we have $|T_{m-1}| = (m-1)k$ and the proof follows by iterating the previous arguments. $\indent \blacksquare$

\begin{lemma}\label{NonEmptyLemma}Suppose there exists a map $\pi: \mathcal{S} \to {[\mathbb{Z}/m\mathbb{Z}] \choose k}$ such that for $k' \in \{r, r+1\}$,
\begin{align}\label{PiSltS}
|\cap_{\ell = 1}^{k'}\pi(S_{i_\ell})| \leq |\cap_{\ell=1}^{k'} S_{i_\ell} |
\end{align}
%
for any set of distinct $i_1, \ldots, i_{k'} \in [m]$. Then $\pi$ is injective and $|\pi(S_v) \cap \cdots \cap \pi(S_{v+(r-1)})| = k-(r-1)$ for all $v \in \mathbb{Z}/m\mathbb{Z}$.
\end{lemma}


\begin{lemma}
Suppose that $m \geq 2k-1$ and there is a function $\pi: \{S_0, \ldots, S_{m-1}\} \to {\mathbb Z/m\mathbb Z \choose k}$ such that for $k' \in \{k, k+1\}$,
\[ \bigcap_{i=1}^{k} S_{i_j} = \emptyset \Longrightarrow \bigcap_{i=1}^{k} \pi(S_{i_j}) = \emptyset.\]
Then $\pi$ is injective and we have:
\[ |S_{i_1} \cap S_{i_2}| = 1  \Longrightarrow |\pi(S_{i_1}) \cap \pi(S_{i_2})| = 1.\]
\end{lemma}


[We need to make a table for small $k$, $m$ of the .]

Let $G$ be a $k$-uniform hypergraph on $m$ nodes (each edge has exactly $k$ elements).
What is the smallest collection of edges 

\begin{problem}
Find the smallest family $\mathcal{F} = \{S_j: j \in J\}$ of $k$-element subsets of $\mathbb Z/mZ$ having the proeprty that for all $v \in \mathbb Z/mZ$, we have
\[\{v\} = \bigcap_{i \in I} S_i, \ \ \text{for some} \ I \subseteq J.\]
\end{problem}


\begin{lemma}
Given a family above of size $|\mathcal{F}|$, one can find $N = k{m \choose k}|\mathcal{F}|$ $k$-sparse $\mathbf{a}_1, \ldots, \mathbf{a}_N$ such that for any $A$ satisfying the spark condition, the dataset $Y = \{A\mathbf{a}_1,\ldots, A\mathbf{a}_N\}$ has a unique sparse coding.
\end{lemma}



\end{document}