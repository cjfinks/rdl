% Notes:
%
% Can we prove in a line or two that the inverse problem is stable by arguing that the function is bijective and closed therefore a homeomorphism?
% Should we mention the notion of Grassmannian manifolds and that Theta is a metric on this space?
% Change robustness to stability?
%
\documentclass[journal, twocolumn]{IEEEtran}

% *** MATH PACKAGES ***
\usepackage{amsmath, amssymb, amsthm} 
\newtheorem{theorem}{Theorem}
\newtheorem{lemma}{Lemma}
\newtheorem{conjecture}{Conjecture}
\newtheorem{problem}{Problem}
\newtheorem{question}{Question}
\newtheorem{proposition}{Proposition}
\newtheorem{definition}{Definition}
\newtheorem{corollary}{Corollary}
\newtheorem{remark}{Remark}
\newtheorem{example}{Example}


\usepackage[pdftex]{graphicx}

% *** ALIGNMENT PACKAGES ***
\usepackage{array}

% correct bad hyphenation here
%\hyphenation{op-tical net-works semi-conduc-tor}

\begin{document}

\title{Sparse Coding is Stable}

\author{Charles~J.~Garfinkle and Christopher~J.~Hillar
\thanks{This research was conducted at the Redwood Center for Theoretical Neuroscience, Berkeley, CA, USA; e-mails: cjg@berkeley.edu, chillar@msri.org.  Support was provided, in part, by National Science Foundation grants IIS-1219212 (CH), IIS-1219199 (CG), and a SAMSI Working Group (CH, CG).}}

\maketitle

\begin{abstract}
Sparse coding has exposed underlying structure in many kinds of natural data.  However, given the multitude of algorithms implementing this strategy, claims of ``true" latent discovery require the backing of universal theorems guaranteeing statistical consistency.  Here we prove that for almost all diverse enough datasets generated under this model, sparse coding to within a constant multiple of measurement error uniquely identifies original codes and dictionary up to uniform permutation and scaling ambiguities.  Applications are given to smoothed analysis, neural communication theory, and engineering.
%Sparse coding or dictionary learning methods have exposed underlying structure in many kinds of natural data. Here, we generalize to the case of measurement noise previous results guaranteeing when the learned dictionary and sparse codes are unique up to inherent permutation and scaling ambiguities.  Central to our proofs is a useful lemma in combinatorial matrix theory allowing us to derive bounds on the number of samples necessary to guarantee uniqueness. We also provide probabilistic extensions to our robust identifiability theorem and an extension to the case when only an upper bound on the number of dictionary elements is known a priori. Our results help to inform the interpretation of sparse structure learned from data; whenever the conditions to one of our robust identifiability theorems are met, any sparsity-constrained algorithm that succeeds in approximately reconstructing the data well enough recovers the original dictionary and sparse codes up to an error commensurate with the noise. Applications are given to smoothed analysis, communication theory, and applied physics.
\end{abstract}

\begin{IEEEkeywords}
Bilinear inverse problem, matrix factorization, identifiability, dictionary learning, sparse coding, compressed sensing, blind source separation, sparse component analysis
\end{IEEEkeywords}

%===================================
% 			INTRODUCTION
%===================================

% Mention application to data compression
% Also I should translate epsilon into angle, i.e. the norm in the matrix space into the metric in the space of subspaces.
\section{Introduction}
\IEEEPARstart{E}{ver} since sparse coding of natural images reproduced response properties of neurons in mammalian primary visual cortex \cite{Olshausen96}, learning sparse representations of vector-valued data has become an important component of many signal processing and machine learning applications (see \cite{Zhang15} for a comprehensive review). In the sparse coding model, each vector in a dataset $Y = \{\mathbf{y}_1, \ldots, \mathbf{y}_N\} \subset \mathbb{R}^n$ is approximated as a linear combination of few vectors drawn from a learned \emph{dictionary} $\mathcal{A} \subset \mathbb{R}^n$, where $n \leq m:= |\mathcal{A}| \ll N$. 

Many algorithms have been designed to infer the underlying parameters of this model, and it is tempting to accept their output as approximating ``ground truth" (e.g., animal position \cite{Agarwal14}) when it exists. However, it is possible that several qualitatively different solutions are nonetheless consistent with the data. It is therefore important to determine algorithm-independent conditions guaranteeing when the sparse representation of a dataset is uniquely identifiable up to natural symmetries and also stable with respect to noise. 
%a dictionary $\mathcal{A}$ of a given size can be uniquely identified up to natural symmetries from a dataset $Y$ and when each datum, in turn, has a unique sparse decomposition into elements of this dictionary. 

The significant finding of this work is that any dictionary satisfying the spark condition from compressed sensing (CS) is uniquely identifiable from enough sparse noisy linear combinations of its elements up to an error linear in the noise (Thm.~\ref{DeterministicUniquenessTheorem}).
%We note that this is a necessary condition if the dataset $Y$ could in principle include data generated as in \eqref{LinearModel} from any given $k$-sparse vector $\mathbf{x} \in \mathbb{R}^m$. 
% [ *** Can we motivate more necessary conditions using theorems of Li15, too? Thm.~2.8. *** ] 
%We also characterize the stability of these solutions with respect to noise, an essential consideration as measurements are inevitably rendered uncertain by various sources of error.
The explicit criteria under which this result holds can serve as a theoretical tool in the analysis of sparse coding routines, some of which now provably converge to a global solution when it exists (see \cite[Sec.~I-E]{Sun16} for a brief discussion of the state-of-the-art in these algorithms). 

Several related problems seem to fall under the umbrella term of sparse coding (also ``dictionary learning" or ``sparse component analysis"). We consider the following formulation, which we believe best represents the original motivating philosophy. Fix a dictionary to be the columns $A_j$ of a matrix $A \in \mathbb R^{n \times m}$. Suppose  $Y$ consists of measurements:
\begin{align}\label{LinearModel}
\mathbf{y} = A\mathbf{a} + \mathbf{n},
\end{align}
for $k$-\emph{sparse} $\mathbf{a} \in \mathbb{R}^m$ having at most $k$ nonzero entries and \emph{noise} $\mathbf{n} \in \mathbb{R}^n$, with $|\mathbf{n}|_2 \leq \varepsilon$. The noise represents our combined uncertainty in  measuring $A\mathbf{a}$.
%; if these uncertainties are all bounded then, since they sum linearly, the resulting total is also bounded, i.e. we suppose for all data in $Y$ that $|\mathbf{n}|_2 \leq \varepsilon$ for some $\varepsilon \geq 0$. 

\begin{problem}[Sparse Coding]\label{InverseProblem}
Find $B \in \mathbb{R}^{n \times m}$ and $k$-sparse $\mathbf{b}_1, \ldots, \mathbf{b}_N \in \mathbb{R}^m$ such that $|\mathbf{y}_i - B\mathbf{b}_i|_2 \leq \varepsilon$ for $i = 1, \ldots, N$.
\end{problem}

Evidently for uniqueness to hold -- knowing only that the $\mathbf{a}_i$ are $k$-sparse -- we require that $A$ satisfy the \textit{spark condition}:
\begin{align}\label{SparkCondition}
A\mathbf{x}_1 = A\mathbf{x}_2 \implies \mathbf{x}_1 = \mathbf{x}_2,\indent \text{for all $k$-sparse } \mathbf{x}_1, \mathbf{x}_2.
\end{align}
Even this not enough to guarantee uniqueness in Problem \ref{InverseProblem}, however, since any $B = AD^{-1}P^{T}$ and $\mathbf{b}_i = PD\mathbf{a}_i$ also works for any permutation matrix $P \in \mathbb{R}^{m \times m}$ and invertible diagonal matrix $D \in \mathbb{R}^{m \times m}$. When $\varepsilon = 0$, this permutation-scaling ambiguity is known to be the only obstruction to uniqueness for sufficiently large $N$ \cite{li2004analysis, Georgiev05, Aharon06, Hillar15}. Matrices of the form $PD$ thus form the \emph{ambiguity transformation group} inherent to the noiseless problem for large enough $N$ \cite{Li15}. We introduce the following criterion to handle the general case.
% Actually the main finding are the specific bounds. Necessary conditions can be derived from the BIP paper...
%This is because any dictionary $\mathcal{D}$ generates infinitely many distinct but equally expressive dictionaries by scaling the elements of $\mathcal{D}$ by nonzero factors; hence, the data in $Y$ could have been generated by any one of these alternate dictionaries. This scaling ambiguity is inherent to all bilinear inverse problems \cite{Choudhary14}.  An additional permutation ambiguity derives from the arbitrary ordering mapping elem`ents of $\mathcal{D}$ to columns of $A$.  
%Equivalently, one could interpret the dictionary learning problem as one of identifying a set of elements in a space which equates all points in $\mathbb{R}^{n} \setminus \{\mathbf{0}\}$ which are nonzero scalings of one another (i.e. the set of lines through the origin), in which case the uniqueness of a solution retains its usual meaning. 
% Pretty version: BIPs should all be rephrased in projective space rather than redefine uniqueness?
% yes:  S_m \times \mathbb (P^1(\mathbb R))^m
%If $\varepsilon = 0$, though, then solutions to Problem \ref{InverseProblem} are unique up to this permutation-scaling ambiguity  

\begin{definition}\label{Uniqueness}
Given $A \in \mathbb{R}^{n \times m}$ and $k$-sparse $\mathbf{a}_1, \ldots, \mathbf{a}_N \in \mathbb{R}^m$, 
dataset $Y = \{\mathbf{y}_1, \ldots, \mathbf{y}_N\} \subset \mathbb{R}^n$ with $\mathbf{y}_i = A\mathbf{a}_i$ for $i = 1, \ldots, N$ is called a \textbf{$k$-sparse representation in $\mathbb{R}^m$}. This representation is \textbf{stable} if for every $\delta_1, \delta_2 \geq 0$, there exists $\varepsilon = \varepsilon(\delta_1, \delta_2) \geq 0$ (with $\varepsilon > 0$ when  $\delta_1, \delta_2 > 0$) such that if any $B \in \mathbb{R}^{n \times m}$ and $k$-sparse $\mathbf{b}_1, \ldots, \mathbf{b}_N \in \mathbb{R}^m$ satisfy:
\begin{align}\label{y-Bb}
|\mathbf{y}_i - B\mathbf{b}_i|_2 \leq \varepsilon,\indent \text{for } i = 1, \ldots, N,
\end{align}
%
then there is a permutation matrix $P \in \mathbb{R}^{m \times m}$ and invertible diagonal matrix $D \in \mathbb{R}^{m \times m}$ such that for all $i,j$:
\begin{align}\label{def1}
|A_j - (BPD)_j|_2 \leq \delta_1 \ \ \text{and} \ \ |\mathbf{a}_i - D^{-1}P^{\top}\mathbf{b}_i|_1 \leq \delta_2.
\end{align}
\end{definition}

\begin{question}\label{DUTproblem}
Let $Y = \{\mathbf{y}_1, \ldots, \mathbf{y}_N \} \subset \mathbb{R}^n$ be generated as $\mathbf{y}_i = A\mathbf{a}_i$ for some matrix $A \in \mathbb{R}^{n \times m}$ and $k$-sparse $\mathbf{a}_i \in \mathbb{R}^m$. When does $Y$ have a stable $k$-sparse representation in $\mathbb{R}^m$?
\end{question}

To see how Question \ref{DUTproblem} informs the interpretation of solutions to Problem \ref{InverseProblem}, suppose $Y$ is a stable $k$-sparse representation in $\mathbb{R}^m$. Fix $\delta_1, \delta_2 \geq 0$, the desired accuracy in uniqueness, and let $\varepsilon = \varepsilon(\delta_1, \delta_2)$ be as in Def.~\ref{Uniqueness}. 
Then, any algorithm solving Problem~\ref{InverseProblem} has solution satisfying \eqref{def1} for some $P$, $D$. 
%That is, the dictionary and codes are uniquely determined to within $\delta_1$ and $\delta_2$, respectively, up to the inherent permutation-scaling ambiguity %and an error commensurate with noise.

Our main result (Thm.~\ref{DeterministicUniquenessTheorem}), stated in Sec.~\ref{Results}, is that Question \ref{DUTproblem} has a positive answer when $A$ satisfies (\ref{SparkCondition}).  A feature of our argument is an explicit form for $\varepsilon(\delta_1, \delta_2)$.  We also provide a probabilistic version of the result (Thm.~\ref{ProbabilisticTheorem}) and an extension to the case where only an upper bound on the dimensionality of the sparse codes (or, equivalently, the number of columns in the generating dictionary) is known (Thm.~\ref{DeterministicUniquenessTheorem2}). We prove Thm.~\ref{DeterministicUniquenessTheorem} in Sec.~\ref{DUT} after listing some additional definitions and stating key lemmas required for the proof, including our main tool from combinatorial matrix theory (Lem.~\ref{MainLemma}). All other proofs are relegated to the supplementary materials. The final section contains several applications of our results.

%In Section \ref{mleqm}, we state another extension of this result (Thm.~\ref{DeterministicUniquenessTheorem2}) to the case where only an upper bound on the dimensionality of the sparse codes (or, equivalently, the number of columns in the generating dictionary) is known. 
%In Section \ref{PUT}, we state Thm.~\ref{ProbabilisticTheorem}, our probabilistic version of Thm.~\ref{DeterministicUniquenessTheorem}. 
%Supplementary materials contain proofs of Cor.~\ref{DeterministicUniquenessCorollary}, Thms.~\ref{DeterministicUniquenessTheorem2}~and~\ref{ProbabilisticTheorem} and our main tool, Lem.~\ref{MainLemma}.

\section{Results}\label{Results}

Before precisely stating our results, we explain how the spark condition relates to the \emph{lower bound} \cite{Grcar10} of $A$, written $L(A)$, which is the largest number $\alpha$ such that $|A\mathbf{x}|_2 \geq \alpha|\mathbf{x}|_2$ for all $\mathbf{x} \in \mathbb{R}^m$. By compactness, every injective linear map has a nonzero lower bound; hence, if $A$ satisfies the spark condition, then every submatrix formed from $2k$ of its columns or less has a nonzero lower bound. We therefore define the following domain-restricted lower bound of $A$:
\begin{align*}
L_k(A) := \max \{ \alpha : |A\mathbf{x}|_2 \geq \alpha|\mathbf{x}|_2 \text{ for all $k$-sparse } \mathbf{x} \in \mathbb{R}^m\}.
\end{align*} 
Clearly, $L_k(A) \geq L_{k'}(A)$ whenever $k < k'$, and for any $A$ satisfying \eqref{SparkCondition}, we have $L_{k'}(A) > 0$ for all $k' \leq 2k$. 

A \textit{cyclic order} on $[m] := \{1, \ldots,m\}$ is an arrangement of $[m]$ in a circular necklace, and an \textit{interval} in the order is any subset of contiguous elements. A vector $\mathbf{x} \in \mathbb{R}^m$ is said to be \emph{supported} on $S \subseteq [m]$ when $\mathbf{x} \in \text{Span}\{ \{\mathbf{e}_i\}_{i\in S}\}$, where $\mathbf{e}_i$ are the standard basis vectors.  Also, recall that $M_j$ denotes the $j$th column of a matrix $M$. The following is our main result.

%=== STATEMENT OF DETERMINISTIC UNIQUENESS THEOREM ===%
\begin{theorem}\label{DeterministicUniquenessTheorem}
Fix $n, m$, and $k < m$. If $A \in \mathbb{R}^{n \times m}$ satisfies spark condition \eqref{SparkCondition} and $k$-sparse \mbox{$\mathbf{a}_1, \ldots, \mathbf{a}_N \in \mathbb{R}^m$} are such that for every interval of length $k$ in some cyclic order on $[m]$ there are at least \mbox{$(k-1){m \choose k}+1$} vectors $\mathbf{a}_i$ in general linear position supported there, then $Y = \{A\mathbf{a}_1, \ldots, A\mathbf{a}_N\}$ has a stable $k$-sparse representation in $\mathbb{R}^m$.

Specifically, there exists a constant $C > 0$ for which the following holds for all $\varepsilon < \frac{L_2(A)}{\sqrt{2}}C^{-1}$. If \textbf{any} matrix $B \in \mathbb{R}^{n \times m}$ and $k$-sparse $\mathbf{b}_1, \ldots, \mathbf{b}_N \in \mathbb{R}^m$ are such that \mbox{$|A\mathbf{a}_i - B\mathbf{b}_i|_2 \leq \varepsilon$} for all $i \in [N]$, then for all $j \in [m]$:
\begin{align}\label{Cstable}
|A_j-(BPD)_j|_2 \leq C\varepsilon,
\end{align}
%
for some permutation matrix $P \in \mathbb{R}^{m \times m}$ and invertible diagonal matrix $D \in \mathbb{R}^{m \times m}$.  Moreover, if $\varepsilon < \varepsilon_0 := \frac{L_{2k}(A)}{\sqrt{2k}}C^{-1}$, then $B$ also satisfies the spark condition and for all $i \in [N]$:
\begin{align}\label{b-PDa}
|\mathbf{a}_i - D^{-1}P^{-1}\mathbf{b}_i|_1 &\leq \frac{\varepsilon }{ \varepsilon_0 - \varepsilon} \left( C^{-1}+|\mathbf{a}_i|_1 \right).
\end{align}
\end{theorem}

\begin{remark}
Note that it was not assumed a priori that $B$ satisfy the spark condition. In fact, when $\varepsilon < \varepsilon_0$ we have:
\begin{align}
L_{2k}(BPD) \geq L_{2k}(A)\left( 1 - \frac{\varepsilon}{\varepsilon_0} \right).
\end{align}
\end{remark}

% Can we get rid of the k-sparse assumption on b using results from CS?

%\begin{remark}[Support Properties]
%Note that the bound in \eqref{b-PDa} does not immediately entail that $\mathbf{a}_i$ and $D^{-1}P^{-1}\mathbf{b}_i$ share the same support unless $\varepsilon = 0$. If $\varepsilon$ is small enough, however, then it is indeed the case that $\text{supp}(D^{-1}P^{-1}\mathbf{b}_i) \subseteq \text{supp}(\mathbf{a}_i)$. [*** TODO: Is this right? See Donoho paper on L1 stability? ***]
% Kinf of like m' > m scenario
%\end{remark}

%[ *** So inference is stable for small enough error! There is no multiplicity of alternative models that also happen to work. Regression models have a fixed $A$ (i.e. the values of the variables we are regressing on). Here, in this model, we don't specify the $A$, instead we learn what the best regressor variables would be if they existed. And we want to combine regressors for different problems so as to save resources. Thought of all this reading section 8 of the Breiman paper on data models vs. algorithmic models.***]

An important consequence of this result is that for sufficiently small reconstruction error, the original dictionary and $k$-sparse codes are determined up to a commensurate error. Specifically, for fixed $\delta_1, \delta_2 \geq 0$, Thm.~\ref{DeterministicUniquenessTheorem} says that \eqref{y-Bb} implies \eqref{def1} for any $\varepsilon < \varepsilon_0$ satisfying:
\begin{align*}
\varepsilon \leq \min \left( \delta_1 C^{-1}, \frac{ \delta_2 \varepsilon_0}{\delta_2 + C^{-1} + \max_{i \in [N]} |\mathbf{a}_i|_1} \right).
\end{align*}
The constant $C = C(Y)$ is explicitly defined in \eqref{Cdef}, below. 

\begin{corollary}\label{DeterministicUniquenessCorollary}
Given $n, m$, and $k < m$, there are $N =  m(k-1){m \choose k}+m$ vectors \mbox{$\mathbf{a}_1, \ldots, \mathbf{a}_N \in \mathbb{R}^m$} with the following property: every matrix $A \in \mathbb{R}^{n \times m}$ satisfying \eqref{SparkCondition} generates a robustly identifiable dataset \mbox{$Y = \{A\mathbf{a}_1, \ldots, A\mathbf{a}_N\}$}.
\end{corollary}

Our proof of Thm.~\ref{DeterministicUniquenessTheorem} in Sec.~\ref{DUT} is a refinement of the arguments in \cite{Hillar15} to handle noise and reduce the number of required samples from $N=k{m \choose k}^2$ to $N = m(k-1){m \choose k}+m$. 

It is straightforward to provide a probabilistic extension of Thm.~\ref{DeterministicUniquenessTheorem} by drawing on the following key result in random matrix theory. A random matrix $A \in \mathbb{R}^{n \times m}$ satisfies \eqref{SparkCondition} with probability one 
%(or ``high probability" for discrete variables) 
provided:
\begin{align}\label{CScondition}
n \geq \gamma k\log\left(\frac{m}{k}\right),
\end{align}
where $\gamma$ is a positive constant dependent on the particular distribution from which $A$ is sampled (many ensembles suffice, e.g. \cite[Sec.~4]{Baraniuk08}). 
[*** mention algebraic goem problem --  ask Bernd ***]

\begin{theorem}\label{ProbabilisticTheorem}
Fix $n, m$ and $k$ satisfying \eqref{CScondition} and let the coefficients of the matrix $A \in \mathbb{R}^{n \times m}$ and $k$-sparse vectors $\mathbf{a}_1, \ldots, \mathbf{a}_N \in \mathbb{R}^m$ be drawn independently from probability measures which are absolutely continuous with respect to the standard Borel measure $\mu$. If for every interval of length $k$ in some cyclic order on $[m]$ there are at least $(k-1){m \choose k} + 1$ vectors $\mathbf{a}_i$ supported on that interval, then with probability one the dataset $Y = \{A\mathbf{a}_1, \ldots, A\mathbf{a}_N\}$ has a stable $k$-sparse representation in $\mathbb{R}^m$. 
\end{theorem}

We note that our result in the deterministic case (Thm.~\ref{DeterministicUniquenessTheorem}) accounts for \emph{worst case} noise.  However, for fixed sparsity $k$, the larger the ambient dimension $n$ of the data, the smaller the probability that noise points in a direction that confuses signals generated by $k$ columns of $A$.  Therefore, for a given distribution, the ``effective'' noise might be much smaller, with the original dictionary and sparse codes being identifiable for better constants with high probability. 

We next state a version of Thm.~\ref{DeterministicUniquenessTheorem} that addresses dictionary recovery when only an upper bound $m'$ on the latent dimension $m$ is known. To do so, we must assume that $B$ also satisfies the spark condition. 

\begin{theorem}\label{DeterministicUniquenessTheorem2}
Let $A \in \mathbb{R}^{n \times m}$ and the $\mathbf{a}_i \in \mathbb{R}^m$ be as described in Thm.~\ref{DeterministicUniquenessTheorem}, and fix $m' > m$. If $B \in \mathbb{R}^{n \times m'}$ also satisfies the spark condition and \mbox{$|A\mathbf{a}_i - B\mathbf{b}_i|_2 \leq \varepsilon$} for all $i \in [N]$ with $k$-sparse $\mathbf{b}_i \in \mathbb{R}^{m'}$, then there exists a constant $C > 0$ such that \eqref{Cstable} and \eqref{b-PDa} hold with respect to a \textbf{partial} permutation matrix $P \in \mathbb{R}^{m' \times m'}$ (i.e., there is at most one nonzero entry in each row and column, and these nonzero entries are all one) and a \textbf{diagonal} matrix $D \in \mathbb{R}^{m' \times m}$ (i.e., $D_{ij} = 0$ when $i \neq j$). 
\end{theorem}

In other words, the columns of $B$ contain (up to noise, after appropriate scaling) the columns of the original dictionary $A$ and the coefficients in each $\mathbf{a}_i$ form (up to noise) a scaled subset of the coefficients in $\mathbf{b}_i$. The constant \mbox{$C = C(A, \{\mathbf{a}_i\}_{i=1}^N)$} in this case is given in the proof of the theorem. 


\section{Proof of Theorem \ref{DeterministicUniquenessTheorem}}\label{DUT}
%===================================
% 			Preliminaries
%===================================
Before proving Theorem \ref{DeterministicUniquenessTheorem}, we briefly outline our main tools, which include general notions of angle (Def.~\ref{FriedrichsDefinition}) and distance (Def.~\ref{GapMetricDef}) between subspaces as well as a (stable) uniqueness result in matrix theory (Lem.~\ref{MainLemma}).
Let ${[m] \choose k}$ be all subsets of $[m]$ of size $k$, and let $\text{Span}\{\mathbf{v}_1, \ldots, \mathbf{v}_\ell\}$ be the $\mathbb{R}$-linear span of vectors $\mathbf{v}_1, \ldots, \mathbf{v}_\ell$.  For a matrix $M$, the spectral norm is denoted $\|M\|_2$.
Also, given $S \subseteq [m]$ and $M \in \mathbb{R}^{n \times m}$, let $M_S$ be the submatrix with columns $M_j$ for $j \in S$.  
%Whenever appropriate we write 
%, and also set $\text{Span}\{M_S\} := \text{Span}\{M_j : j \in S\}$.  

Between any pair of subspaces in Euclidean space one can define the following generalized ``angle''.
\begin{definition}\label{FriedrichsDefinition}
The \textbf{Friedrichs angle} $\theta_F = \theta_F(U,V) \in [0,\frac{\pi}{2}]$ between subspaces $U,V \subseteq \mathbb{R}^n$ is defined in terms of its cosine:
\begin{align*}
\cos{\theta_F} := \max\left\{ \langle u, v \rangle: \substack{ u \in U \cap (U \cap V)^\perp \cap \mathcal{B} \\ v \in V \cap (U \cap V)^\perp \cap \mathcal{B} } \right\},
\end{align*}
where $\mathcal{B} = \{ x: |x|_2 \leq 1\}$ is the unit $\ell_2$-ball in $\mathbb{R}^n$ \cite{Deutsch12}.
\end{definition}
For example, when $n=3$ and $k=1$, this is simply the angle between vectors; and for $k=2$, it is the angle between the normal vectors of two planes. In higher dimensions, the Friedrichs angle is one out of a set of \textit{principal} (or \textit{canonical} or \textit{Jordan}) angles between subspaces that are invariant to orthogonal transformations. These angles are all zero if and only if one subspace is a subset of the other; otherwise, the Friedrichs angle is the smallest nonzero such angle. 

The next quantity is based on one used in \cite{Deutsch12} to analyze the convergence of the alternating projections algorithm for projecting a point onto the intersection of a set of subspaces.
% We use it to bound the distance between a point and the intersection of a set of subspaces given an upper bound on the distance from that point to each individual subspace. 

%The minimization over permutations in \eqref{xidef} below is done only to remove the dependence the convergence result has on the order in which subspaces are inputted to the alternating projections algorithm.

\begin{definition}\label{SpecialSupportSet}
Fix $A \in \mathbb{R}^{n \times m}$ and $k < m$. Setting $\phi_1(A) := 1$, define for $k \geq 2$:
\begin{align*}
\phi_k(A) := \min_{ S_1,\ldots,S_k \in {[m] \choose k} } 1 - \xi( \text{\rm Span}\{A_{S_1}\}, \ldots,  \text{\rm Span}\{A_{S_k}\}),
\end{align*}
where for any set $\mathcal{V} = \{V_1, \ldots, V_k\}$ of subspaces of $\mathbb{R}^m$, 
\begin{align*}
\xi(\mathcal{V}) := \min_{\sigma \in \frak{S}_k} \left(1 - \prod_{i=1}^{k-1} \sin^2  \theta_F \left(V_{\sigma(i)}, \cap_{j=i+1}^k V_j \right)  \right)^{1/2},
\end{align*}
and $\frak{S}_{k}$ are the permutations (bijections) on $k$ elements. 
\end{definition}

%=== SPECIFICS OF DETERMINISTIC THEOREM ===%
We are now in a position to state explicitly the constant $C$ referred to in Thm. ~\ref{DeterministicUniquenessTheorem}. Letting $T$ be the set of supports on which the $\mathbf{a}_i$ are supported (intervals of length $k$ in some cyclic ordering of $[m]$), $X$ be the $m \times N$ matrix with columns $\mathbf{a}_i$, and $I(S) := \{i : S = \text{supp}(\mathbf{a}_i)\}$, we have:
\begin{align}\label{Cdef}
C = \left( \frac{ \sqrt{k^3}}{ \phi_k(A) } \right) \frac{\max_{j \in [m]} |A_j|_2}{\min_{S \in T} L_k(AX_{I(S)})}.
\end{align}

\begin{remark}\label{nonzero}
We can be sure that $C$ is well-defined provided $\min_{S \in T} L_k(AX_{I(S)}) > 0$, since $\phi_k(A) = 0$ only when $\text{Span}(A_{S_1}) \supseteq \text{Span}(A_{S_2}) \cap \cdots \cap \text{Span}(A_{S_k})$ for some $S_1, \ldots, S_k \in {[m] \choose k}$, which would be in violation of \eqref{SparkCondition}.
\end{remark}

\begin{definition}\label{GapMetricDef}
Let $U, V$ be subspaces of $\mathbb{R}^m$ and let $d(u,V) := \inf\{|u-v|_2: v \in V\} = |u - \Pi_V u|_2$, where $\Pi_V$ is the orthogonal projection operator onto subspace $V$. The \textbf{gap metric} $\Theta$ is defined as \cite{Akhiezer13}:
\begin{equation*}
\Theta(U,V) := \max\left( \sup_{\substack{u \in U \\ |u|_2 = 1}} d(u,V), \sup_{\substack{v \in V \\ |v|_2 = 1}} d(v,U) \right).
\end{equation*}
\end{definition}

In fact, $\Theta(U,V)$ is equal to the sine of the largest Jordan angle between $U$ and $V$. 

We now state our main result from combinatorial matrix theory, generalizing \cite[Lem.~1]{Hillar15} to the noisy case.

%===========          MAIN LEMMA (K > 1)             =================
\begin{lemma}[Main Lemma]\label{MainLemma}
Fix $n, m$, $k < m$, and let $T$ be the set of intervals of length $k$ in some cyclic ordering of $[m]$. Let $A, B \in \mathbb{R}^{n \times m}$ and suppose that $A$ satisfies the spark condition \eqref{SparkCondition} and has maximum column $\ell_2$-norm $\rho$.  If there exists a map $\pi: T \to {[m] \choose k}$ and some $\delta < \frac{L_{2}(A)}{\sqrt{2}}$ such that 
\begin{equation}\label{GapUpperBound}
\Theta(\text{\rm Span}\{A_{S}\}, \text{\rm Span}\{B_{\pi(S)}\}) \leq \frac{ \phi_k(A) }{\rho k} \delta, \indent \text{for } S \in T,
\end{equation}
%
then there exist a permutation matrix $P \in \mathbb{R}^{m \times m}$ and an invertible diagonal matrix $D \in \mathbb{R}^{m \times m}$ with
\begin{align*}
|A_j - (BPD)_j|_2 \leq \delta, \indent \text{for } j \in [m].
\end{align*}
\end{lemma}

We will also use the following useful facts about the distance $d$ from Def.~\ref{GapMetricDef}. The first, 
\begin{equation}\label{SubspaceMetricSameDim}
\dim(W) = \dim(V) \implies \sup_{\substack{v \in V \\ |v|_2 = 1}}  d(v,W)  = \sup_{\substack{w \in W \\ |w|_2 = 1}} d(w,V),
\end{equation}
can be found in \cite[Lem.~3.3]{Morris10}. The second is:
\begin{lemma}\label{MinDimLemma}
If $U, V$ are subspaces of $\mathbb{R}^{m}$, then
\begin{equation*}
d(u,V) < |u|_2, \ \ u \in U \setminus{\{0\}} \implies \dim(U) \leq \dim(V).
\end{equation*}
\end{lemma}

%===================================
% 			Deterministic Identifiability
%===================================
%======== CASE K = 1 ============
%Let $\mathbf{e}_1, \ldots, \mathbf{e}_m$ be the standard basis vectors in $\mathbb R^m$.
Let us now prove Thm.~\ref{DeterministicUniquenessTheorem} for the simple case when $k=1$. Fix $A \in \mathbb{R}^{n \times m}$ satisfying spark condition \eqref{SparkCondition} and suppose we have $N = m$ $1$-sparse vectors $\mathbf{a}_j = c_j \mathbf{e}_j$ for $c_j \in \mathbb{R} \setminus \{0\}$, $j \in [m]$. By \eqref{Cdef}, we have:
\begin{align}\label{C1}
C 
%&= \left( \frac{ \sqrt{k^3}}{ \phi_1(A) } \right) \frac{\max_{j \in [m]} |A_j|_2}{\min_{i \in [m]} L_1(c_iA_i) } \\
&= \sqrt{k^3} \left( \frac{\max_{j \in [m]} |A_j|_2}{\min_{i \in [m]}|c_iA_i|_2} \right)
\geq \max_{i \in [m]} \frac{1}{|c_i|}.
\end{align}

Suppose that for some $B \in \mathbb{R}^{n \times m}$ and 1-sparse $\mathbf{b}_i \in \mathbb{R}^m$ we have  $|A\mathbf{a}_i - B\mathbf{b}_i|_2 \leq \varepsilon < \frac{L_2(A)}{\sqrt{2}}C^{-1}$ for $i \in [m]$. Since the $\mathbf{b}_i$ are 1-sparse, there must exist $c'_1, \ldots, c'_m \in \mathbb{R}$ and some map $\pi: [m] \to [m]$ such that 
\begin{align}\label{1D}
|c_iA_i - c'_iB_{\pi(i)}|_2 \leq \varepsilon, \ \ \text{for } \  i \in [m].
\end{align} 
Note that $c'_i \neq 0$ for all $i$ since then otherwise (by definition of $L_2(A)$), we reach the contradiction $|c_iA_i|_2 < \min_{i \in [m]}|c_iA_i|_2$. 

We will now show that $\pi$ is necessarily injective (and thus defines a permutation). Suppose that $\pi(i) = \pi(j) = \ell$ for some $i \neq j$ and $\ell \in [m]$. Then, $|c_iA_i - c'_iB_{\ell}|_2  \leq \varepsilon$ and $|c_jA_j - c'_jB_{\ell}|_2 \leq \varepsilon$. Scaling and summing these inequalities by $|c'_j|$ and $|c'_i|$, respectively, and applying the triangle inequality, we have:
\begin{align}\label{contra}
(|c'_i| + |c'_j|) \varepsilon
&\geq |A(c'_jc_i\mathbf{e}_i - c'_ic_j\mathbf{e}_j)|_2 \nonumber \\ 
&\geq \frac{L_2(A)}{\sqrt{2}} \left( |c'_j| + |c'_i| \right) \min_{\ell \in [m]} |c_\ell |,
\end{align}
%
where the last inequality follows from the definition of $L_2(A)$ and the fact that $|\mathbf{x}|_1 \leq \sqrt{k}|\mathbf{x}|_2$ for $k$-sparse $\mathbf{x}$. Since \eqref{contra} is in contradiction with \eqref{C1} and our upper bound on $\varepsilon$, it must be that $\pi$ is in fact injective. Letting $P = \left( \mathbf{e}_{\pi(1)} \cdots \mathbf{e}_{\pi(m)}\right)$ and $D = \text{diag}(\frac{c'_1}{c_1},\ldots,\frac{c'_m}{c_m})$, we see that \eqref{1D} becomes for $i \in [m]$:
\begin{align}\label{k=1result}
|A_i - (BPD)_i|_2 = |A_i - \frac{c'_i}{c_i}B_{\pi(i)}|_2 \leq \frac{\varepsilon}{|c_i|} \leq C\varepsilon.
\end{align}

% ======== b - PDa =========
\begin{remark}\label{b-PDaProof}
It is enough to know \eqref{k=1result} in order to bound $|\mathbf{a}_i - D^{-1}P^{-1}\mathbf{b}_i|_1$ as well. Specifically, \eqref{b-PDa} always follows from \eqref{Cstable} when $\varepsilon < \varepsilon_0 := \frac{L_{2k}(A)}{\sqrt{2k}}C^{-1}$. To see why, note that for all $2k$-sparse $\mathbf{x} \in \mathbb{R}^m$ we have $|(A-BPD)\mathbf{x}|_2 
\leq C\varepsilon|\mathbf{x}|_1
\leq C \varepsilon \sqrt{2k}  |\mathbf{x}|_2.$
by the triangle inequality. Thus,
\begin{align*}
|BPD\mathbf{x}|_2 
&\geq | |A\mathbf{x}|_2 - |(A-BPD)\mathbf{x}|_2 | \\
&\geq (L_{2k}(A) - \sqrt{2k}C\varepsilon ) |\mathbf{x}|_2,
\end{align*}
%
where in the last inequality we drop the absolute value since $\varepsilon < \varepsilon_0$. Hence, $L_{2k}(BPD) \geq L_{2k}(A)\left( 1 - \varepsilon/\varepsilon_0 \right) > 0$ and \eqref{b-PDa} follows from a straightforward calculation.
%\begin{align*}
%|D^{-1}P^{-1}\mathbf{b}_i - \mathbf{a}_i|_1
%&\leq \sqrt{2k} |\mathbf{a}_i - D^{-1}P^{-1}\mathbf{b}_i|_2 \\
%&\leq \frac{\sqrt{2k}}{L_{2k}(BPD)}|BPD(\mathbf{a}_i - D^{-1}P^{-1}\mathbf{b}_i)|_2 \\
%&\leq \frac{\sqrt{2k}}{L_{2k}(BPD)} (|B\mathbf{b}_i - A\mathbf{a}_i|_2 + |(A - BPD)\mathbf{a}_i|_2) \\
%&\leq \frac{\varepsilon\sqrt{2k}}{L_{2k}(BPD)}(1+C|\mathbf{a}_i|_1) \\
%&\leq \frac{\varepsilon }{\varepsilon_0 - \varepsilon} \left( C^{-1}+|\mathbf{a}_i|_1 \right).
%\end{align*}
\end{remark}

% ========== DEFINITIONS FOR MAIN LEMMA (K>1) ================
It remains to show that \eqref{Cstable} with $C$ given in \eqref{Cdef} follows from $\varepsilon < \frac{L_2(A)}{\sqrt{2}}C^{-1}$ for $k > 1$. Our main tool is Lem.~\ref{MainLemma}.

%========          PROOF OF THEOREM 1        ============
\begin{proof}[Proof of Thm.~\ref{DeterministicUniquenessTheorem}]
Let $T$ be the set of intervals of length $k$ in the given cyclic order of $[m]$. From above, we may assume that $k > 1$. Fix $N = m(k-1){m \choose k}+m$ vectors in $\mathbb{R}^k$ as in the statement of the theorem. Fix $A \in \mathbb{R}^{n \times m}$ satisfying \eqref{SparkCondition}. We claim that $\{A\mathbf{a}_1, \ldots, A\mathbf{a}_N\}$ has a robustly identifiable $k$-sparse representation in $\mathbb{R}^m$. Suppose that for some $B \in \mathbb{R}^{n \times m}$ there exist $k$-sparse $\mathbf{b}_i \in \mathbb{R}^m$ such that $|A\mathbf{a}_i - B\mathbf{b}_i|_2 \leq \varepsilon$ for all $i \in [N]$. Since there are $(k-1){m \choose k}+1$ vectors $\mathbf{a}_i$ with a given support $S \in T$, the pigeon-hole principle implies that there exists some $S' \in {[m] \choose k}$ and some set of indices $J(S)$ of cardinality $k$ such that all $\mathbf{a}_i$ and $\mathbf{b}_i$ with $i \in J(S)$ have supports $S$ and $S'$, respectively.

Let $X$ and $X'$ be the $m \times N$ matrices with columns $\mathbf{a}_i$ and $\mathbf{b}_i$, respectively. It follows from the general linear position of the $\mathbf{a}_i$ and the linear independence of every $k$ columns of $A$ that the columns of the $n \times k$ matrix $AX_{J(S)}$ are linearly independent, i.e. $L(AX_{J(S)}) > 0$, and therefore form a basis for $\text{Span}\{A_{S}\}$. Fixing $\mathbf{z} \in \text{Span}\{A_{S}\}$, there then exists a unique $\mathbf{c} = (c_1, \ldots, c_k) \in \mathbb{R}^k$ such that $\mathbf{z} = AX_{J(S)}\mathbf{c}$. Letting $\mathbf{z'} = BX'_{J(S)}\mathbf{c}$, which is in $\text{Span}\{B_{S'}\}$, we have:
\begin{align*}
|\mathbf{z} - \mathbf{z'}|_2 
&= |\sum_{j=1}^N c_i(AX_{J(S)} - BX'_{J(S)})\mathbf{e}_j|_2 
\leq \varepsilon \sum_{j=1}^N |c_j| \\
&\leq \varepsilon \sqrt{k}  |\mathbf{c}|_2 
\leq \frac{\varepsilon \sqrt{k}}{L(AX_{J(S)})} |AX_{J(S)}\mathbf{c}|_2 \\
&= \frac{\varepsilon \sqrt{k}}{L(AX_{J(S)})} |\mathbf{z}|_2.
\end{align*}

Hence,
\begin{align}\label{ABSubspaceDistance}
\sup_{ \substack{ \mathbf{z} \in \text{Span}\{A_{S}\} \\ |\mathbf{z}|_2 = 1} } d(\mathbf{z}, \text{Span}\{B_{S'}\}) \leq \frac{\varepsilon\sqrt{k}}{L(AX_{J(S)})}.
\end{align}

We now show that \eqref{Cstable} follows if $\varepsilon < \frac{L_2(A)}{\sqrt{2}}C^{-1}$, with $C$ as defined in \eqref{Cdef}. In this case we can bound the RHS of \eqref {ABSubspaceDistance} as follows. Letting $\rho = \max_{j \in [m]} |A_j|_2$ and $I(S) = \{i: \text{supp}(\mathbf{a}_i)=S\}$, we have:
\begin{align}\label{rhs}
\frac{\varepsilon\sqrt{k}}{L(AX_{J(S)})} 
&<  \frac{\phi_k(A) L_2(A)}{\rho k \sqrt{2}} \left( \frac{\min_{S \in T}L_k(AX_{I(S)})}{L(AX_{J(S)})} \right) \nonumber \\
&\leq \frac{\phi_k(A)}{\rho k} \left( \frac{L_2(A)}{\sqrt{2}} \right).
\end{align}

Since $L_2(A) \leq \rho \sqrt{2}$ and $\phi_k(A) \leq 1$, we have that the RHS of \eqref{ABSubspaceDistance} is strictly less than one. It follows by Lem.~\ref{MinDimLemma} that $\dim(\text{Span}\{B_{S'}\}) \geq \dim(\text{Span}\{A_{S}\}) = k$ (since every $k$ columns of $A$ are linearly independent). Since $|S'| = k$, we have $\dim(\text{Span}\{B_{S'}\}) \leq k$; hence, $\dim(\text{Span}\{B_{S'}\}) = \dim(\text{Span}\{A_{S}\})$. Recalling \eqref{SubspaceMetricSameDim},  we see the association $S \mapsto S'$ thus defines a map $\pi: T \to {[m] \choose k}$ satisfying
\begin{align}\label{yeyeye}
\Theta(\text{Span}\{A_{S}\}, \text{Span}\{B_{\pi(S)}\}) \leq \frac{\varepsilon\sqrt{k}}{L(AX_{J(S)})} \indent \text{for } S \in T.
\end{align}

From \eqref{rhs} and \eqref{yeyeye} we see that the inequality $\Theta(\text{Span}\{A_{S}\}, \text{Span}\{B_{\pi(S)}\}) \leq \frac{ \phi_k(A) }{\rho k} \delta$ is satisfied for $\delta < \frac{L_2(A)}{\sqrt{2}}$ by setting $\delta = \frac{ \rho k}{ \phi_k(A) } \left(  \frac{\varepsilon \sqrt{k}}{L(AX_{J(S)})} \right)$ (see Rem.~\ref{nonzero} for why $\phi_k(A) \neq 0$). We therefore satisfy \eqref{GapUpperBound} for 
\begin{align*}
\delta = \left( \frac{ \varepsilon \sqrt{k^3}}{ \phi_k(A) } \right) \frac{\max_{j \in [m]} |A_j|_2}{\min_{S \in T} L_k(AX_{I(S)})}
= C\varepsilon.
\end{align*}

It follows by Lem.~\ref{MainLemma} that there exist a permutation matrix $P \in \mathbb{R}^{m \times m}$ and invertible diagonal matrix $D \in \mathbb{R}^{m \times m}$ such that $|A_j - (BPD)_j|_2 \leq C\varepsilon$ for all $j \in [m]$. The proof of how \eqref{b-PDa} follows from this result is contained in Rem.~\ref{b-PDaProof}.
\end{proof}


%===================================
% DIFFERENT CODING DIMENSIONS
%===================================


%===================================
% 			DISCUSSION
%===================================
\section{Discussion}

In this note, we generalize the known uniqueness of solutions to \eqref{InverseProblem} in the exact case ($\varepsilon = 0$) to the more realistic case of deterministic noise ($\varepsilon > 0$).  Somewhat surprisingly, as long as a standard assumption from compressed sensing is met, almost every dictionary and sufficient quantity $N$ of sparse codes are uniquely determined up to the error in sampling and inherent symmetries (uniform relabeling and scaling). To convince the reader of the general usefulness of this result, we elaborate briefly on four diverse application areas.

%\textbf{Compressed Sensing}.
%The goal of compressed sensing is to recover a signal $\mathbf{x} \in \mathbb{R}^n$ that is sparse enough in some known basis (i.e., $\mathbf{a} = \Psi^{-1} \mathbf{x}$ is $k$-sparse for some invertible $\Psi$) via a stable and efficient reconstruction process after it has been linearly subsampled as $\mathbf{y} = \Phi \mathbf{x} + \mathbf{n}$ by a known compression matrix $\Phi \in \mathbb{R}^{n \times m}$ and noise $\mathbf{n}$. If $A = \Phi\Psi$ satisfies the spark condition \eqref{SparkCondition}, then $\mathbf{a}$ is identifiable given $\mathbf{y}$ and the signal $\mathbf{x}$ can then be reconstructed as $\Psi \mathbf{a}$. But what if $\Phi$ is unknown? Our theorems show that the matrix $A$ and sparse codes can still be estimated up to noise and an inherent permutation-scaling ambiguity if samples $\mathbf{y}_1, \ldots, \mathbf{y}_N$ are sufficiently diverse. 

\textbf{Data Analysis}.  
In sparse coding, it is usually assumed that the linear model \eqref{LinearModel} describes some physical process, and the goal is to infer the true underlying dictionary and sparse codes from measurements. Often in such analyses, it is implicitly assumed that sparse coding has exposed this underlying sparse structure in the data as opposed to some artificial degenerate solution arising from particular implementation details of an algorithm. Our results provide theoretical backing to claims that given enough data samples uniqueness is indeed the norm rather than the exception. Regarding this, it would be useful to determine for general $(m,n,k)$ the best possible dependence of $\varepsilon$ on $\delta_1, \delta_2$ (see Def. \ref{Uniqueness}) as well as the minimum possible sample size $N$ and ``diversity" of generating codes. We encourage the sparse coding community to extend our results and find a tight dependency on all the parameters, both for the sake of theory and practical applications.

\textbf{Smoothed Analysis}.
The main concept in smoothed analysis \cite{Spielman04} is that certain algorithms having bad worst case behavior are, nonetheless, efficient if certain (typically, measure zero in the continuous case and with ``low probability" in the discrete case) pathological input sets are avoided. Our results imply that if there is an efficient ``smoothed" algorithm for solving Problem \ref{InverseProblem} given enough samples, then for generic inputs this algorithm determines the unique original solution. We note that avoiding ``bad" (NP-hard) sets of inputs is necessary for dictionary learning \cite{Razaviyayn15, Tillmann15}.

\textbf{Neural Communication Theory}.
In \cite{Coulter10} and \cite{Isely10}, it was posited that sparse features of natural data passed through a communication bottleneck in the brain using random projections could be decoded by unsupervised sparse coding.  A necessary condition for this theory to work is that the sparse coding problem has a unique solution.  This was already verified in the case of data sampled without noise.  This work extends this theory to the more realistic case of sampling error.

\textbf{Engineering}.
Several groups have found ways to utilize compressed sensing (CS) for signal processing tasks, such as digital image compression \cite{Duarte08} (``single-pixel camera") and, more recently, the design of an ultrafast camera \cite{Gao14}. Given such effective uses of classical CS, it is only a matter of time before these systems utilize sparse coding algorithms to code and process data. In this case, guarantees such as the ones offered by our theorems allow any such device to be ambiguity-transform equivalent to any other (having different initial parameters and data samples) as long as enough data originates from a statistically equivalent system.
% MENTION THE PEOPLE AT CARNEGIE MELON WHO CONTACTED US?

%===================================
% 		ACKNOWLEDGEMENT
%===================================
\section*{Acknowledgment}
We thank Fritz Sommer for turning our attention to the sparse coding problem, Darren Rhea for sharing early explorations, and Ian Morris for posting identity \eqref{SubspaceMetricSameDim} with a reference to his proof on the internet (www.stackexchange.com). %Finally, we thank Bizzyskillet of Soudcloud.com for the ``No Exam Jams'', which played on repeat during many long hours of designing proofs.

%===================================
% 			REFERENCES
%===================================
\bibliographystyle{IEEEtran}
\bibliography{chazthm_ieee}

%===================================
% 			BIOGRAPHY
%===================================
\begin{IEEEbiographynophoto}{Charles J. Garfinkle}
completed a B.S. in Physics and Chemistry at McGill University. He is currently a Ph.D. candidate in Neuroscience at UC Berkeley.
\end{IEEEbiographynophoto}

\begin{IEEEbiographynophoto}{Christopher J. Hillar}
graduated with a B.S. from Yale University in 2000 and received a Ph.D. in Mathematics from the University of California (UC), Berkeley in 2005, supported by an NSF Graduate Research Fellowship.  From 2005-2008, he was an NSF Postdoctoral Fellow at Texas A\&M University. From 2008-2010, he was an NSF All Institutes Postdoctoral Fellow at the Mathematical Sciences Research Institute in Berkeley, CA.  In 2010, he joined the Redwood Center for Theoretical Neuroscience at UC Berkeley.  % , and in 2011, began working part time in psychiatry department at UC San Francisco.
\end{IEEEbiographynophoto}

\clearpage
\section{SUPPLEMENTAL MATERIAL}

Thm.~\ref{DeterministicUniquenessTheorem2} is easily proven by modification of the arguments in Sec.~\ref{DUT}. In the case $k=1$, we simply set the codomain of $\pi$ to be $[m']$ instead of $[m]$, which then defines the partial permutation $P = \left( \mathbf{e}_{\pi(1)} \cdots \mathbf{e}_{\pi(m)}, \mathbf{0}, \ldots, \mathbf{0} \right) \in \mathbb{R}^{m' \times m'}$, while $D = \text{diag}(\frac{c'_1}{c_1},\ldots,\frac{c'_m}{c_m}) \in \mathbb{R}^{m' \times m}$. The proof for general $k>1$ is contained in the supplementary material, and is simpler as a result of the additional assumption on $B$. Finally, the manipulations in Rem.~\ref{b-PDaProof} can similarly be modified to show that \eqref{b-PDa} follows from \eqref{Cstable}. 

\begin{remark}
We demonstrate with the following counter-example that for $C$ as defined in \eqref{Cdef} the condition $\varepsilon < \frac{L_2(A)}{\sqrt{2}}C^{-1}$ is necessary to guarantee in general that \eqref{Cstable} follows from the remaining assumptions of Theorem \ref{DeterministicUniquenessTheorem}. Consider the dataset $\mathbf{a}_i = \mathbf{e}_i$ for $i = 1, \ldots, m$ and let $A$ be the identity matrix in $\mathbb{R}^{m \times m}$. Then $L_2(A) = 1$ (we have $|A\mathbf{x}|_2 = |\mathbf{x}|_2$ for all $\mathbf{x} \in \mathbb{R}^m$) and $C = 1$; hence $\frac{L_2(A)}{\sqrt{2}}C^{-1} = 1/\sqrt{2}$. Consider the alternate dictionary $B = \left(\mathbf{0}, \frac{1}{2}(\mathbf{e}_1 + \mathbf{e}_2), \mathbf{e}_3, \ldots, \mathbf{e}_{m} \right)$ and sparse codes $\mathbf{b}_i = \mathbf{e}_2$ for $i = 1, 2$ and $\mathbf{b}_i = \mathbf{e}_i$ for $i = 3, \ldots, m$. Then $|A\mathbf{a}_i - B\mathbf{b}_i| = 1/\sqrt{2}$ for $i = 1, 2$ (and $0$ otherwise). If there were permutation and invertible diagonal matrices $P \in \mathbb{R}^{m \times m}$ and $D \in \mathbb{R}^{m \times m}$ such that $|(A-BPD)\mathbf{e}_i| \leq C\varepsilon$ for all $i \in [m]$, then we would reach the contradiction $1 = |P^{-1}\mathbf{e}_1|_2 = |(A-BPD)P^{-1}\mathbf{e}_1|_2 \leq 1/\sqrt{2}$. 
\end{remark}

\begin{proof}[Proof of Corollary \ref{DeterministicUniquenessCorollary}]
The first step is to produce a set of $N = m(k-1){m \choose k}+m$ vectors in $\mathbb{R}^k$ in general linear position (i.e., any $k$ of them are linearly independent).  Specifically, let $\gamma_1, \ldots, \gamma_N$ be any distinct numbers. Then the columns of the $k \times N$ matrix $V = (\gamma^i_j)^{k,N}_{i,j=1}$ are in general linear position (since the $\gamma_j$ are distinct, any $k \times k$ ``Vandermonde" sub-determinant is nonzero). Next, form the $k$-sparse vectors $\mathbf{a}_1, \ldots, \mathbf{a}_N \in \mathbb{R}^m$ with supports $S \in T$ (partitioning the $a_i$ evenly among these supports so that each support contains $(k-1){m \choose k}+1$ vectors $a_i$) by setting the nonzero values of vector $\mathbf{a}_i$ to be those contained in the $i$th column of $V$.
\end{proof}

\begin{proof}[Proof of Lemma \ref{MinDimLemma}]
We prove the contrapositive.  If $\dim(U) > \dim(V)$, then a dimension argument ($\dim U + \dim V^\perp > m$) gives a nonzero $u \in U \cap V^\perp$.  In particular, we have $|u - v|_2^2 = |u|_2^2 + |v|_2^2 \geq |u|_2^2$ for $v \in V$, and thus $d(u,V) \geq |u|_2$.
\end{proof}
% PUT THIS IN APPENIX?


\appendices
\section{Combinatorial Matrix Theory}\label{appendixA}

In this section, we prove Lemma \ref{MainLemma}, which is the main ingredient in our proof of Theorem \ref{DeterministicUniquenessTheorem}. For readers willing to assume a priori that the spark condition holds for $B$ as well as for $A$, a shorter proof of this case (Lemma \ref{MainLemma2} from Sec.~\ref{mleqm}) is provided in Appendix \ref{mleqmAppendix}. This additional assumption simplifies the argument and allows us to extend robust identifiability conditions to the case where only an upper bound on the number of columns $m$ in $A$ is known. 

We now prove some auxiliary lemmas before deriving Lemma \ref{MainLemma}.  Given sets $\mathcal{T}$, let $\cap \mathcal{T}$ denote their intersection.

%===== SPAN INTERSECTION LEMMA =====
\begin{lemma}\label{SpanIntersectionLemma}
Let $M \in \mathbb{R}^{n \times m}$. If every $2k$ columns of $M$ are linearly independent, then for any $\mathcal{T} \subseteq \bigcup_{\ell \leq k} {[m] \choose \ell}$,
\begin{align}
\text{\rm Span}\{M_{\cap \mathcal{T}}\}  = \bigcap_{S \in \mathcal{T}} \text{\rm Span}\{M_S\}.
\end{align}
\end{lemma}

\begin{proof}By induction, it is enough to prove the lemma when $|\mathcal{T}| = 2$. The proof now follows directly from the assumption.
\end{proof}

%===== DISTANCE TO INTERSECTION LEMMA =====

\begin{lemma}\label{DistanceToIntersectionLemma}
Fix $k \geq 2$. Let $\mathcal{V} = \{V_1, \ldots, V_k\}$ be subspaces of $\mathbb{R}^m$ and let $V = \bigcap \mathcal{V}$. For every $\mathbf{x} \in \mathbb{R}^m$, we have
\begin{align}\label{DTILeq}
|\mathbf{x} - \Pi_V \mathbf{x}|_2 \leq \frac{1}{1 - \xi(\mathcal{V})} \sum_{i=1}^k |x - \Pi_{V_i} x|_2,
\end{align}
provided $\xi(\mathcal{V}) \neq 1$, where $\xi$ is given in Def.~\ref{SpecialSupportSet}.
\end{lemma}
\begin{proof} 
Fix $\mathbf{x} \in \mathbb{R}^m$ and $k \geq 2$. The proof consists of two parts. First, we shall show that 
\begin{equation}\label{induction}
|\mathbf{x} - \Pi_V\mathbf{x}|_2 \leq \sum_{\ell=1}^k |\mathbf{x} - \Pi_{V_{\ell}} \mathbf{x}|_2 + |\Pi_{V_{k}}\Pi_{V_{k-1}}\cdots\Pi_{V_{1}} \mathbf{x} - \Pi_V \mathbf{x}|_2.
\end{equation}
For each $\ell \in \{2, \ldots, k+1\}$ (when $\ell = k+1$, the product $\Pi_{V_k} \cdots \Pi_{V_{\ell}}$ is set to $I$), we have by the triangle inequality and the fact that $\|\Pi_{V_{\ell}}\|_2 \leq 1$ (as $\Pi_{V_{\ell}}$ are projections):
\begin{equation}
|\Pi_{V_k} \cdots \Pi_{V_{\ell}}\mathbf{x} - \Pi_V \mathbf{x}|  \leq  |\Pi_{V_k} \cdots \Pi_{V_{\ell-1}}\mathbf{x} - \Pi_V \mathbf{x}| + 
|\mathbf{x} - \Pi_{V_{\ell-1}}\mathbf{x}|.
\end{equation}
Summing these inequalities over $\ell$ gives (\ref{induction}).

Next, we show how the result \eqref{DTILeq} follows from \eqref{induction} from the following result of \cite[Theorem 9.33]{Deutsch12}:
\begin{align}\label{dti2}
|\Pi_{V_k}\Pi_{V_{k-1}}\cdots\Pi_{V_1} \mathbf{x} - \Pi_V\mathbf{x}|_2 \leq z |\mathbf{x}|_2 \indent \text{for } \indent \mathbf{x} \in \mathbb{R}^m,
\end{align}
where $z= \left[1 - \prod_{\ell =1}^{k-1}(1-z_{\ell}^2)\right]^{1/2}$ and $z_{\ell} = \cos\theta_F\left(V_{\ell}, \cap_{s=\ell+1}^k V_s\right)$. To see this, note that
\begin{align}\label{dti1}
|\Pi_{V_k}\Pi_{V_{k-1}}\cdots\Pi_{V_1}(\mathbf{x} - \Pi_V\mathbf{x}) - \Pi_V(\mathbf{x} - \Pi_V\mathbf{x})|_2& \\
= |\Pi_{V_k}\Pi_{V_{k-1}}\cdots\Pi_{V_1} \mathbf{x} - \Pi_V \mathbf{x} |_2&,
\end{align}
since $\Pi_{V_\ell} \Pi_V = \Pi_V$ for all $\ell = 1, \ldots, k$ and $\Pi_V^2 = \Pi_V$.
Therefore by \eqref{dti2} and \eqref{dti1}, it follows that
\begin{align*}
|\Pi_{V_k}&\Pi_{V_{k-1}}\cdots\Pi_{V_1} \mathbf{x} - \Pi_V \mathbf{x} |_2 \\
&= |\Pi_{V_k}\Pi_{V_{k-1}}\cdots\Pi_{V_1}(\mathbf{x} - \Pi_V\mathbf{x}) - \Pi_V(\mathbf{x} - \Pi_V\mathbf{x})|_2 \\
&\leq z |\mathbf{x} - \Pi_V\mathbf{x}|_2.
\end{align*}
Combining this last inequality with \eqref{induction} and rearranging, we arrive at
\begin{align}\label{ceq}
|\mathbf{x} - \Pi_V \mathbf{x}|_2 \leq \frac{1}{1 - z} \sum_{i=1}^k |\mathbf{x} - \Pi_{V_i} \mathbf{x}|_2.
\end{align}
Finally, since the ordering of the subspaces is arbitrary, we can replace $z$ in \eqref{ceq} with $\xi(\mathcal{V})$ to obtain \eqref{DTILeq}.
\end{proof}

%======= GRAPH THEORY LEMMA =======

\begin{lemma}\label{NonEmptyLemma} Fix integers $k < m$, and let $T = \{S_1, \ldots, S_m\}$ be the set of contiguous length $k$ intervals in some cyclic order of $[m]$. Suppose there exists a map $\pi: T \to {[m] \choose k}$ such that
\begin{align}\label{NonEmpty}
|\bigcap_{i \in J} \pi(S_i)| \leq |\bigcap_{i \in J} S_i | \ \ \text{for } \ J \in {[m] \choose k}.
\end{align}
%
Then, $|\pi(S_{j_1}) \cap \cdots \cap \pi(S_{j_k})| = 1$ for $j_1,\ldots,j_k$ consecutive modulo $m$.
\end{lemma}

\begin{proof} Consider the set $Q_m = \{ (r,t) : r \in \pi(S_t), t \in [m] \}$, which has $mk$ elements. By the pigeon-hole principle, there is some $q \in [m]$ and $J \in {[m] \choose k}$ such that $(q, j) \in Q_m$ for all $j \in J$. In particular, we have $q \in \cap_{j \in J} \pi(S_j)$ so that from \eqref{NonEmpty} there must be some $p \in [m]$ with $p \in \cap_{j \in J} S_j$. Since $|J| = k$, this is only possible if the elements of $J = \{j_1, \ldots, j_k\}$ are consecutive modulo $m$, in which case $|\cap_{j \in J} S_j| = 1$. Hence $|\cap_{j \in J} \pi(S_j)| = 1$ as well.

We next consider if any other $t \notin J$ is such that $q \in \pi(S_t)$. Suppose there were such a $t$; then, we would have $q \in \pi(S_t) \cap \pi(S_{j_1}) \cap \cdots \cap \pi(S_{j_k})$ and \eqref{NonEmpty} would imply that the intersection of every $k$-element subset of $\{S_t\} \cup \{S_j: j \in J\}$ is nonempty. This would only be possible if $\{t\} \cup J = [m]$, in which case the result then trivially holds since then $q \in \pi(S_j)$ for all $j \in [m]$.  Suppose now there exists no such $t$; then letting $Q_{m-1} \subset Q_m$ be the set of elements of $Q_m$ not having $q$ as a first coordinate, we have $|Q_{m-1}| = (m-1)k$. 

By iterating the above arguments we arrive at a partitioning of $Q_m$ into sets $R_i = Q_i \setminus Q_{i-1}$ for $i = 1, \ldots, m$, each having a unique element of $[m]$ as a first coordinate common to all $k$ elements while having second coordinates which form a consecutive set modulo $m$. In fact, every set of $k$ consecutive integers modulo $m$ is the set of second coordinates of some $R_i$. This must be the case because for every consecutive set $J$ we have $|\cap_{j \in J} S_j| = 1$, whereas if $J$ is the set of second coordinates for two distinct sets $R_i$ we would have $|\cap_{j \in J} \pi(S_j)| \geq 2$, which violates \eqref{NonEmpty}. 
\end{proof}

%==== PROOF OF MAIN LEMMA =======
\begin{proof}[Proof of Lemma \ref{MainLemma} (Main Lemma)]
We assume $k \geq 2$ since the case $k = 1$ was proven at the beginning of Sec.~\ref{DUT}. Let $S_1, \ldots, S_m$ be the set of contiguous length $k$ intervals in some cyclic ordering of $[m]$. We begin by proving that $\dim(\text{Span}\{B_{\pi(S_i)}\}) = k$ for all $i \in [m]$. 
Fix $i \in [m]$ and note that by \eqref{GapUpperBound} we have for all unit vectors $\mathbf{u} \in \text{Span}\{A_{S_i}\}$ that $d(u, \text{Span}\{B_{\pi(S_i)}\}) \leq \frac{\phi_k(A)}{\rho k} \delta$ for $\delta < \frac{L_2(A)}{ \sqrt{2}}$. By definition of $L_2(A)$ we have for all $2$-sparse $\mathbf{x} \in \mathbb{R}^m$:
\begin{align}
L_2(A) \leq \frac{|A\mathbf{x}|_2}{|\mathbf{x}|_2} \leq \rho \frac{|\mathbf{x}|_1}{|\mathbf{x}|_2} \leq \rho \sqrt{2}
\end{align}

Hence $\delta < \rho$. Since $\phi_k \leq 1$ we have $d(u, \text{Span}\{B_{\pi(S_i)}\}) < 1$ and it follows by Lemma \ref{MinDimLemma} that $\dim(\text{Span}\{B_{\pi(S_i)}\}) \geq \dim(\text{Span}\{A_{S_i}\}) = k$. Since $|\pi(S_i)| = k$, we in fact have $\dim(\text{Span}\{B_{\pi(S_i)}\}) = k$. %, i.e. the columns of $B_{\pi(S_\sigma(i))}$ are linearly independent. 

We will now show that
\begin{align}\label{fact2}
|\bigcap_{i \in J} \pi(S_i)| \leq |\bigcap_{i \in J} S_i | \ \ \text{for } \ J \in {[m] \choose k}.
\end{align}

Fix $J \in {[m] \choose k}$. By \eqref{GapUpperBound} we have for all unit vectors $\mathbf{u} \in \cap_{i \in J} \text{Span}\{B_{\pi(S_i)}\}$ that $d(\mathbf{u}, \text{Span}\{A_{S_i}\}) \leq \frac{\phi_k(A)}{\rho k} \delta$ for all $j \in J$, where $\delta < \frac{L_2(A)}{\sqrt{2}}$. It follows by Lemma \ref{DistanceToIntersectionLemma} that
\begin{align*}
d\left( \mathbf{u}, \bigcap_{i \in J} \text{Span}\{A_{S_j}\} \right) 
\leq \frac{\delta}{\rho} \left( \frac{ \phi_k(A) }{1 - \xi( \{ \text{Span}\{A_{S_i}\}: i \in J\} ) } \right) \leq \frac{\delta}{\rho},
\end{align*}
%
where the second inequality follows immediately from the definition of $\phi_k(A)$. 

Now, since \mbox{$\text{Span}\{B_{\cap_{i \in J}\pi(S_i)}\} \subseteq \cap_{i \in J} \text{Span}\{B_{\pi(S_i)}\}$} and (by Lemma \ref{SpanIntersectionLemma}) $\cap_{i \in J}  \text{Span}\{A_{S_i}\} = \text{Span}\{A_{\cap_{i \in J}  S_i}\}$, we have
\begin{align}\label{fact1}
d\left( \mathbf{u}, \text{Span}\{A_{\cap_{i \in J} S_i}\} \right) \leq \frac{\delta}{\rho} \indent \text{for unit vectors } \mathbf{u} \in \text{Span}\{B_{\cap_{i \in J}\pi(S_i)}\}.
\end{align}
We therefore have by Lemma \ref{MinDimLemma} (since $\delta/\rho < 1$) that $\dim(\text{Span}\{B_{\cap_{i \in J}\pi(S_i)}\}) \leq \dim(\text{Span}\{A_{\cap_{i \in J} S_i}\})$ and \eqref{fact2} follows by the linear independence of the columns of $A_{S_i}$ and $B_{\pi(S_i)}$ for all $i \in [m]$.

Suppose now that $J = \{i-k+1, \ldots, i\}$ so that $\cap_{i \in J} S_i = i$. By \eqref{fact2} we have that $\cap_{i \in J} \pi(S_i)$ is either empty or it contains a single element. Lemma \ref{NonEmptyLemma} ensures that the latter case is the only possibility. Thus, the association $i \mapsto \cap_{i \in J} \pi(S_i)$ defines a map $\hat \pi: [m] \to [m]$. Recalling \eqref{SubspaceMetricSameDim}, it follows from \eqref{fact1} that for all unit vectors $\mathbf{u} \in \text{Span}\{A_i\}$ we have $d\left( \mathbf{u}, \text{Span}\{B_{\hat \pi(i)}\}\right) \leq \delta/\rho$ also. Since $i$ is arbitrary, it follows that for every canonical basis vector $\mathbf{e}_i \in \mathbb{R}^m$, letting $c_i = |A\mathbf{e}_i|_2^{-1}$ and $\varepsilon = \delta/\rho$, there exists some $c'_i \in \mathbb{R}$ such that $|c_iA\mathbf{e}_i - c'_iB\mathbf{e}_{\hat \pi(i)}|_2 \leq \varepsilon$ where $\varepsilon < \frac{L_2(A)}{\sqrt{2}} \min_{j \in [m]} c_i$. This is exactly the supposition in \eqref{1D} and the result follows from the subsequent arguments of Sec.~\ref{DUT}. 
\end{proof}

\begin{remark}
The arguments above can easily be modified to prove Lemma \ref{MainLemma2}. Since Lemma \ref{NonEmptyLemma} assumes $m = m'$, we may not invoke it when $m' > m$ to show that $|\cap_{i \in J} \pi(S_i)| = 1$ for $J = \{i-k+1, \ldots, i\}$. Instead, under the additional assumption that $B$ satisfies spark condition \eqref{SparkCondition}, we can swap the roles of $A$ and $B$ in the proof of \eqref{fact1} to show that $\dim(\text{Span}\{B_{\cap_{i \in J}\pi(S_i)}\}) = \dim(\text{Span}\{A_{\cap_{i \in J} S_i}\})$, from which the same fact then follows. The proof is then completed in much the same way as above by defining a map $\pi: [m] \to [m']$ by the association $i \mapsto \cap_{i \in J} \pi(S_i)$, thereby reducing the proof to the $k=1$ case described in Rem.~\ref{m'geqmk=1}.
\end{remark}

%\begin{remark} In general, there may exist combinations of fewer supports with intersection $\{i\}$, e.g. if $m \geq 2k-1$ then $S_{i - (k-1)} \cap S_i = \{i\}$. For brevity, we have considered a construction that is valid for any $k < m$.
%\end{remark}

The proof of Thm.~\ref{DeterministicUniquenessTheorem2} is very similar to the proof of Thm.~\ref{DeterministicUniquenessTheorem}, the difference being that now we establish a map $\pi: [m] \to [m']$ satisfying the requirements of Lem.~\ref{MainLemma2}, which we state next, by pigeonholing $(k-1){m' \choose k} + 1$ vectors with respect to holes $[m']$. This insures that we can establish a one-to-one correspondence between subspaces spanned by the $m'$ columns of $B$ and nearby subspaces spanned by the $m$ columns of $A$ despite the fact that $m < m'$. By requiring $B$ to also satisfy the spark condition, we remove the dependency of Lem.~\ref{MainLemma} on Lem.~\ref{NonEmptyLemma} (which requires that $m = m'$), resulting in Lem.~\ref{MainLemma2}.

and for Thm. ~\ref{DeterministicUniquenessTheorem2}:
\begin{align}\label{Cdef2}
C:= \left( \frac{ \sqrt{k^3}}{ \min(\phi_k(A), \phi_k(B)) } \right) \frac{\max_{j \in [m]} |A_j|_2}{\min_{S \in T} L_k(AX_{I(S)})}.
\end{align}


\begin{lemma}[Main Lemma~for $m < m'$]\label{MainLemma2}
Fix $n, m, m'$, and $k$ where $k < m < m'$, and let $T$ be the set of intervals of length $k$ in some cyclic ordering of $[m]$. Let $A \in \mathbb{R}^{n \times m}$ and $B \in \mathbb{R}^{n \times m'}$ both satisfy spark condition \eqref{SparkCondition} with $A$ having maximum column $\ell_2$-norm $\rho$. If there exists a map $\pi: T \to {[m'] \choose k}$ and some $\delta < \frac{L_{2}(A)}{\sqrt{2}}$ such that for $S \in T$:
\begin{equation}\label{GapUpperBound2}
\Theta(\text{\rm Span}\{A_{S}\}, \text{\rm Span}\{B_{\pi(S)}\}) \leq \frac{ \delta }{\rho k} \min(\phi_k(A), \phi_k(B)),
\end{equation}
then there exist a partial permutation matrix $P \in \mathbb{R}^{m' \times m'}$ and a diagonal matrix $D \in \mathbb{R}^{m' \times m}$ such that
\begin{align}
|A_j - (BPD)_j|_2 \leq \delta, \indent \text{for } j \in [m].
\end{align}
\end{lemma}

We defer the proof of this lemma to Appendix B.


\begin{lemma}\label{Hillar15lemma2}
Fix $n, m$, $k < m$, and a matrix $M \in \mathbb{R}^{n \times m}$ satisfying \eqref{SparkCondition}. With probability one, $\{M\mathbf{a}_1, \ldots, M\mathbf{a}_k\}$ are linearly independent when $\mathbf{a}_i$ are random $k$-sparse vectors.
\end{lemma}

\begin{theorem}\label{Theorem2}
Fix $n, m$, $k < m$, and $A \in \mathbb{R}^{n \times m}$ satisfying \eqref{SparkCondition}. If a set of $N = m(k-1){m \choose k}+m$ randomly drawn vectors $\mathbf{a}_1, \ldots, \mathbf{a}_N \in \mathbb{R}^m$ are such that for each interval of length $k$ in some cyclic order on $[m]$ there are $(k-1){m \choose k} + 1$ vectors $\mathbf{a}_i$ supported on that interval then $\{A\mathbf{a}_1, \ldots, A\mathbf{a}_N\}$ has a robustly identifiable $k$-sparse representation with probability one.
\end{theorem}

\begin{proof}
By Lem.~\ref{Hillar15lemma2} (taking $M = I$), with probability one the $\mathbf{a}_i$ are in general linear position. Apply Thm.~\ref{DeterministicUniquenessTheorem}.
\end{proof} 

\begin{corollary}
Suppose $m, n$, and $k$ satisfy inequality \eqref{CScondition}. With probability one, a random $n \times m$ generation matrix $A$ satisfies \eqref{SparkCondition}. Fixing such an $A$, we have with probability one that a dataset $\{A\mathbf{a}_1, \ldots , A\mathbf{a}_N\}$ generated from a random draw of $N = m(k-1){m \choose k}+m$ $k$-sparse vectors $\mathbf{a}_i$, consisting of $(k-1){m \choose k}+1$ samples supported on each interval of length $k$ in some cyclic ordering of $[m]$, has a robustly identifiable $k$-sparse representation in $\mathbb{R}^m$.
\end{corollary}

Note that an \emph{algebraic set} is a solution to a finite set of polynomial equations. 

\begin{theorem}\label{Theorem3}
Fix $k < m$ and $n$ . If $N = m(k-1){m \choose k}+m$ randomly drawn vectors $\mathbf{a}_1, \ldots, \mathbf{a}_N \in \mathbb{R}^m$ are such that for every interval of length $k$ in some cyclic ordering of $[m]$ there are $(k-1){m \choose k}+1$ vectors $\mathbf{a}_i$ supported on that interval, then with probability one the following holds. There is an algebraic set $Z \subset \mathbb{R}^{n \times m}$ of Lebesgue measure zero with the following property: if $A \notin Z$ then $\{A\mathbf{a}_1, \ldots , A\mathbf{a}_N \}$ has a robustly identifiable $k$-sparse representation in $\mathbb{R}^m$.
\end{theorem}

\begin{proof}
By Lem.~\ref{Hillar15lemma2} (setting $M$ to be the identity matrix), with probability one the $\mathbf{a}_i$ are in general linear position. By the same arguments made in the proof of Thm.~3 in \cite{Hillar15}, the set of matrices $A$ that fail to satisfy \eqref{SparkCondition} form an algebraic set of $\mu$-measure zero. Apply Thm.~\ref{DeterministicUniquenessTheorem}.
\end{proof}

\begin{corollary}
Suppose $m, n$, and $k$ obey inequality \eqref{CScondition}.  If $N = m(k-1){m \choose k}+m$ randomly drawn vectors $\mathbf{a}_1, \ldots, \mathbf{a}_N \in \mathbb{R}^m$ are such that for every interval of length $k$ in some cyclic ordering of $[m]$ there are $(k-1){m \choose k}+1$ vectors $\mathbf{a}_i$ supported on that interval then, with probability one, almost every matrix $A \in \mathbb{R}^{n \times m}$ gives a robustly identifiable $Y = \{A\mathbf{a}_1, \ldots , A\mathbf{a}_N \}$.
\end{corollary}


\begin{proof}
In \cite{Hillar15} it was demonstrated how to construct a polynomial in the entries of the matrix $A$ and $k$-sparse vectors $\mathbf{a}_i$ which does not evaluate to zero if and only if $A$ satisfies \eqref{SparkCondition} and  the $\mathbf{a}_i$ which share supports are in general linear position. By [Baraniuk], there is an $A$ which satisfies the spark condition and by the Vandermonde construction there are $\mathbf{a}_i$ in general linear position. By poly trick, the set of zeros of $f$ form a set of $\mu$-measure 0. Hence the set of matrices and vectors which do not satisfy our constraints form a set of $\mu$-measure 0, where $\mu$ is the standard Borel product measure. By Theorem 1, the set of datasets $Y =  \{A\mathbf{a}_1, \ldots, A\mathbf{a}_N\}$ for which the entries of $A$ and the $k$-sparse vectors $\mathbf{a}_i$ are all independently drawn according to probability measures which are absolutely continuous with respect to $\mu$ form a set of $\mu$-measure 0. 



%First, note that if a set of measure spaces $\{(X_i, \Sigma_i, \nu_i)\}_{i=1}^p$ is such that $\nu_i$ is absolutely continuous with respect to $\mu$ for all $i = 1, \ldots, p$, where $\mu$ is the standard Borel measure on $\mathbb{R}$, then the product measure $\prod_{i=1}^p \nu_i$ is absolutely continuous with respect to the standard Borel product measure on $\mathbb{R}^p$. By HS11, the matrices $A \in \mathbb{R}^{n \times m}$ with entries $A_{ij}$ sampled i.i.d. from the uniform distribution on $[0,1]$ which do \emph{not} satisfy the spark condition form a set of Lebesgue measure 0. Therefore, any matrix $A$ with samples drawn according to probability measures $\nu_{ij}$ which are absolutely continuous with respect to $\mu$ satisfy the spark condition with probability one. Also according to HS11, fixing $S \in {[m] \choose k}$, any set of $k$-sparse vectors $\mathbf{a}_1, \ldots, \mathbf{a}_q \in \mathbb{R}^m$ supported on $S$ are in general linear position with probability one. Apply Theorem 1.
\end{proof}


\end{document}
