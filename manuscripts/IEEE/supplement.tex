% Notes:
%
% Can we prove in a line or two that the inverse problem is stable by arguing that the function is bijective and closed therefore a homeomorphism?
% Should we mention the notion of Grassmannian manifolds and that Theta is a metric on this space?
% Change robustness to stability?
%
\documentclass[journal, twocolumn]{IEEEtran}

% *** MATH PACKAGES ***
\usepackage{amsmath, amssymb, amsthm} 
\newtheorem{theorem}{Theorem}
\newtheorem{lemma}{Lemma}
\newtheorem{conjecture}{Conjecture}
\newtheorem{problem}{Problem}
\newtheorem{question}{Question}
\newtheorem{proposition}{Proposition}
\newtheorem{definition}{Definition}
\newtheorem{corollary}{Corollary}
\newtheorem{remark}{Remark}
\newtheorem{example}{Example}


\usepackage[pdftex]{graphicx}

% *** ALIGNMENT PACKAGES ***
\usepackage{array}

% correct bad hyphenation here
%\hyphenation{op-tical net-works semi-conduc-tor}

\begin{document}

\title{Supplemental Materials \\ Uniqueness and stability in sparse coding}

\author{Charles~J.~Garfinkle and Christopher~J.~Hillar
\thanks{The research of Garfinkle and Hillar was conducted while at the Redwood Center for Theoretical Neuroscience, Berkeley, CA, USA; e-mails: cjg@berkeley.edu, chillar@msri.org.  Support was provided, in part, by National Science Foundation grants IIS-1219212 (CH), IIS-1219199 (CG), and a SAMSI Working Group (CH, CG).}}

\maketitle

%========================================
%      	APPENDIX: COMBINATORICS
%========================================


\begin{remark}
We demonstrate with the following counter-example that for $C$ as defined in \eqref{Cdef} the condition $\varepsilon < \frac{L_2(A)}{\sqrt{2}}C^{-1}$ is necessary to guarantee in general that \eqref{Cstable} follows from the remaining assumptions of Theorem \ref{DeterministicUniquenessTheorem}. Consider the dataset $\mathbf{a}_i = \mathbf{e}_i$ for $i = 1, \ldots, m$ and let $A$ be the identity matrix in $\mathbb{R}^{m \times m}$. Then $L_2(A) = 1$ (we have $|A\mathbf{x}|_2 = |\mathbf{x}|_2$ for all $\mathbf{x} \in \mathbb{R}^m$) and $C = 1$; hence $\frac{L_2(A)}{\sqrt{2}}C^{-1} = 1/\sqrt{2}$. Consider the alternate dictionary $B = \left(\mathbf{0}, \frac{1}{2}(\mathbf{e}_1 + \mathbf{e}_2), \mathbf{e}_3, \ldots, \mathbf{e}_{m} \right)$ and sparse codes $\mathbf{b}_i = \mathbf{e}_2$ for $i = 1, 2$ and $\mathbf{b}_i = \mathbf{e}_i$ for $i = 3, \ldots, m$. Then $|A\mathbf{a}_i - B\mathbf{b}_i| = 1/\sqrt{2}$ for $i = 1, 2$ (and $0$ otherwise). If there were permutation and invertible diagonal matrices $P \in \mathbb{R}^{m \times m}$ and $D \in \mathbb{R}^{m \times m}$ such that $|(A-BPD)\mathbf{e}_i| \leq C\varepsilon$ for all $i \in [m]$, then we would reach the contradiction $1 = |P^{-1}\mathbf{e}_1|_2 = |(A-BPD)P^{-1}\mathbf{e}_1|_2 \leq 1/\sqrt{2}$. 
\end{remark}




\appendices
\section{Combinatorial Matrix Theory}\label{appendixA}

In this section, we prove Lemma \ref{MainLemma}, which is the main ingredient in our proof of Theorem \ref{DeterministicUniquenessTheorem}. For readers willing to assume a priori that the spark condition holds for $B$ as well as for $A$, a shorter proof of this case (Lemma \ref{MainLemma2} from Section \ref{mleqm}) is provided in Appendix \ref{mleqmAppendix}. This additional assumption simplifies the argument and allows us to extend robust identifiability conditions to the case where only an upper bound on the number of columns $m$ in $A$ is known. 

We now prove some auxiliary lemmas before deriving Lemma \ref{MainLemma}.  Given a collection of sets $\mathcal{T}$, we let $\cap \mathcal{T}$ denote their intersection.

%===== SPAN INTERSECTION LEMMA =====
\begin{lemma}\label{SpanIntersectionLemma}
Let $M \in \mathbb{R}^{n \times m}$. If every $2k$ columns of $M$ are linearly independent, then for any $\mathcal{T} \subseteq \bigcup_{\ell \leq k} {[m] \choose \ell}$,
\begin{align}
\text{\rm Span}\{M_{\cap \mathcal{T}}\}  = \bigcap_{S \in \mathcal{T}} \text{\rm Span}\{M_S\}.
\end{align}
\end{lemma}

\begin{proof}By induction, it is enough to prove the lemma when $|\mathcal{T}| = 2$. The proof now follows directly from the assumption.
\end{proof}

%===== DISTANCE TO INTERSECTION LEMMA =====

\begin{lemma}\label{DistanceToIntersectionLemma}
Fix $k \geq 2$. Let $\mathcal{V} = \{V_1, \ldots, V_k\}$ be subspaces of $\mathbb{R}^m$ and let $V = \bigcap \mathcal{V}$. For every $\mathbf{x} \in \mathbb{R}^m$, we have
\begin{align}\label{DTILeq}
|\mathbf{x} - \Pi_V \mathbf{x}|_2 \leq \frac{1}{1 - \xi(\mathcal{V})} \sum_{i=1}^k |x - \Pi_{V_i} x|_2,
\end{align}
provided $\xi(\mathcal{V}) \neq 1$, where the expression for $\xi$ is given in Def.~\ref{SpecialSupportSet}.
\end{lemma}
\begin{proof} 
Fix $\mathbf{x} \in \mathbb{R}^m$ and $k \geq 2$. The proof consists of two parts. First, we shall show that 
\begin{equation}\label{induction}
|\mathbf{x} - \Pi_V\mathbf{x}|_2 \leq \sum_{\ell=1}^k |\mathbf{x} - \Pi_{V_{\ell}} \mathbf{x}|_2 + |\Pi_{V_{k}}\Pi_{V_{k-1}}\cdots\Pi_{V_{1}} \mathbf{x} - \Pi_V \mathbf{x}|_2.
\end{equation}
For each $\ell \in \{2, \ldots, k+1\}$ (when $\ell = k+1$, the product $\Pi_{V_k} \cdots \Pi_{V_{\ell}}$ is set to $I$), we have by the triangle inequality and the fact that $\|\Pi_{V_{\ell}}\|_2 \leq 1$ (as $\Pi_{V_{\ell}}$ are projections):
\begin{equation}
|\Pi_{V_k} \cdots \Pi_{V_{\ell}}\mathbf{x} - \Pi_V \mathbf{x}|  \leq  |\Pi_{V_k} \cdots \Pi_{V_{\ell-1}}\mathbf{x} - \Pi_V \mathbf{x}| + 
|\mathbf{x} - \Pi_{V_{\ell-1}}\mathbf{x}|.
\end{equation}
Summing these inequalities over $\ell$ gives (\ref{induction}).

Next, we show how the result \eqref{DTILeq} follows from \eqref{induction} from the following result of \cite[Theorem 9.33]{Deutsch12}:
\begin{align}\label{dti2}
|\Pi_{V_k}\Pi_{V_{k-1}}\cdots\Pi_{V_1} \mathbf{x} - \Pi_V\mathbf{x}|_2 \leq z |\mathbf{x}|_2 \indent \text{for } \indent \mathbf{x} \in \mathbb{R}^m,
\end{align}
where $z= \left[1 - \prod_{\ell =1}^{k-1}(1-z_{\ell}^2)\right]^{1/2}$ and $z_{\ell} = \cos\theta_F\left(V_{\ell}, \cap_{s=\ell+1}^k V_s\right)$. To see this, note that
\begin{align}\label{dti1}
|\Pi_{V_k}\Pi_{V_{k-1}}\cdots\Pi_{V_1}(\mathbf{x} - \Pi_V\mathbf{x}) - \Pi_V(\mathbf{x} - \Pi_V\mathbf{x})|_2& \\
= |\Pi_{V_k}\Pi_{V_{k-1}}\cdots\Pi_{V_1} \mathbf{x} - \Pi_V \mathbf{x} |_2&,
\end{align}
since $\Pi_{V_\ell} \Pi_V = \Pi_V$ for all $\ell = 1, \ldots, k$ and $\Pi_V^2 = \Pi_V$.
Therefore by \eqref{dti2} and \eqref{dti1}, it follows that
\begin{align*}
|\Pi_{V_k}&\Pi_{V_{k-1}}\cdots\Pi_{V_1} \mathbf{x} - \Pi_V \mathbf{x} |_2 \\
&= |\Pi_{V_k}\Pi_{V_{k-1}}\cdots\Pi_{V_1}(\mathbf{x} - \Pi_V\mathbf{x}) - \Pi_V(\mathbf{x} - \Pi_V\mathbf{x})|_2 \\
&\leq z |\mathbf{x} - \Pi_V\mathbf{x}|_2.
\end{align*}
Combining this last inequality with \eqref{induction} and rearranging, we arrive at
\begin{align}\label{ceq}
|\mathbf{x} - \Pi_V \mathbf{x}|_2 \leq \frac{1}{1 - z} \sum_{i=1}^k |\mathbf{x} - \Pi_{V_i} \mathbf{x}|_2.
\end{align}
Finally, since the ordering of the subspaces is arbitrary, we can replace $z$ in \eqref{ceq} with $\xi(\mathcal{V})$ to obtain \eqref{DTILeq}.
\end{proof}

%======= GRAPH THEORY LEMMA =======

\begin{lemma}\label{NonEmptyLemma} Fix positive integers $k < m$, and let $S_1, \ldots, S_m$ be the set of contiguous length $k$ intervals in some cyclic order of $[m]$. Suppose there exists a map $\pi: T \to {[m] \choose k}$ such that
\begin{align}\label{NonEmpty}
|\bigcap_{i \in J} \pi(S_i)| \leq |\bigcap_{i \in J} S_i | \ \ \text{for } \ J \in {[m] \choose k}.
\end{align}
%
Then, $|\pi(S_{j_1}) \cap \cdots \cap \pi(S_{j_k})| = 1$ for $j_1,\ldots,j_k$ consecutive modulo $m$.
\end{lemma}

\begin{proof} Consider the set $Q_m = \{ (r,t) : r \in \pi(S_t), t \in [m] \}$, which has $mk$ elements. By the pigeon-hole principle, there is some $q \in [m]$ and $J \in {[m] \choose k}$ such that $(q, j) \in Q_m$ for all $j \in J$. In particular, we have $q \in \cap_{j \in J} \pi(S_j)$ so that from \eqref{NonEmpty} there must be some $p \in [m]$ with $p \in \cap_{j \in J} S_j$. Since $|J| = k$, this is only possible if the elements of $J = \{j_1, \ldots, j_k\}$ are consecutive modulo $m$, in which case $|\cap_{j \in J} S_j| = 1$. Hence $|\cap_{j \in J} \pi(S_j)| = 1$ as well.

We next consider if any other $t \notin J$ is such that $q \in \pi(S_t)$. Suppose there were such a $t$; then, we would have $q \in \pi(S_t) \cap \pi(S_{j_1}) \cap \cdots \cap \pi(S_{j_k})$ and \eqref{NonEmpty} would imply that the intersection of every $k$-element subset of $\{S_t\} \cup \{S_j: j \in J\}$ is nonempty. This would only be possible if $\{t\} \cup J = [m]$, in which case the result then trivially holds since then $q \in \pi(S_j)$ for all $j \in [m]$.  Suppose now there exists no such $t$; then letting $Q_{m-1} \subset Q_m$ be the set of elements of $Q_m$ not having $q$ as a first coordinate, we have $|Q_{m-1}| = (m-1)k$. 

By iterating the above arguments we arrive at a partitioning of $Q_m$ into sets $R_i = Q_i \setminus Q_{i-1}$ for $i = 1, \ldots, m$, each having a unique element of $[m]$ as a first coordinate common to all $k$ elements while having second coordinates which form a consecutive set modulo $m$. In fact, every set of $k$ consecutive integers modulo $m$ is the set of second coordinates of some $R_i$. This must be the case because for every consecutive set $J$ we have $|\cap_{j \in J} S_j| = 1$, whereas if $J$ is the set of second coordinates for two distinct sets $R_i$ we would have $|\cap_{j \in J} \pi(S_j)| \geq 2$, which violates \eqref{NonEmpty}. 
\end{proof}

%==== PROOF OF MAIN LEMMA =======
\begin{proof}[Proof of Lemma \ref{MainLemma} (Main Lemma)]
We assume $k \geq 2$ since the case $k = 1$ was proven at the beginning of Section \ref{DUT}. Let $S_1, \ldots, S_m$ be the set of contiguous length $k$ intervals in some cyclic ordering of $[m]$. We begin by proving that $\dim(\text{Span}\{B_{\pi(S_i)}\}) = k$ for all $i \in [m]$. 
Fix $i \in [m]$ and note that by \eqref{GapUpperBound} we have for all unit vectors $\mathbf{u} \in \text{Span}\{A_{S_i}\}$ that $d(u, \text{Span}\{B_{\pi(S_i)}\}) \leq \frac{\phi_k(A)}{\rho k} \delta$ for $\delta < \frac{L_2(A)}{ \sqrt{2}}$. By definition of $L_2(A)$ we have for all $2$-sparse $\mathbf{x} \in \mathbb{R}^m$:
\begin{align}
L_2(A) \leq \frac{|A\mathbf{x}|_2}{|\mathbf{x}|_2} \leq \rho \frac{|\mathbf{x}|_1}{|\mathbf{x}|_2} \leq \rho \sqrt{2}
\end{align}

Hence $\delta < \rho$. Since $\phi_k \leq 1$ we have $d(u, \text{Span}\{B_{\pi(S_i)}\}) < 1$ and it follows by Lemma \ref{MinDimLemma} that $\dim(\text{Span}\{B_{\pi(S_i)}\}) \geq \dim(\text{Span}\{A_{S_i}\}) = k$. Since $|\pi(S_i)| = k$, we in fact have $\dim(\text{Span}\{B_{\pi(S_i)}\}) = k$. %, i.e. the columns of $B_{\pi(S_\sigma(i))}$ are linearly independent. 

We will now show that
\begin{align}\label{fact2}
|\bigcap_{i \in J} \pi(S_i)| \leq |\bigcap_{i \in J} S_i | \ \ \text{for } \ J \in {[m] \choose k}.
\end{align}

Fix $J \in {[m] \choose k}$. By \eqref{GapUpperBound} we have for all unit vectors $\mathbf{u} \in \cap_{i \in J} \text{Span}\{B_{\pi(S_i)}\}$ that $d(\mathbf{u}, \text{Span}\{A_{S_i}\}) \leq \frac{\phi_k(A)}{\rho k} \delta$ for all $j \in J$, where $\delta < \frac{L_2(A)}{\sqrt{2}}$. It follows by Lemma \ref{DistanceToIntersectionLemma} that
\begin{align*}
d\left( \mathbf{u}, \bigcap_{i \in J} \text{Span}\{A_{S_j}\} \right) 
\leq \frac{\delta}{\rho} \left( \frac{ \phi_k(A) }{1 - \xi( \{ \text{Span}\{A_{S_i}\}: i \in J\} ) } \right) \leq \frac{\delta}{\rho},
\end{align*}
%
where the second inequality follows immediately from the definition of $\phi_k(A)$. 

Now, since \mbox{$\text{Span}\{B_{\cap_{i \in J}\pi(S_i)}\} \subseteq \cap_{i \in J} \text{Span}\{B_{\pi(S_i)}\}$} and (by Lemma \ref{SpanIntersectionLemma}) $\cap_{i \in J}  \text{Span}\{A_{S_i}\} = \text{Span}\{A_{\cap_{i \in J}  S_i}\}$, we have
\begin{align}\label{fact1}
d\left( \mathbf{u}, \text{Span}\{A_{\cap_{i \in J} S_i}\} \right) \leq \frac{\delta}{\rho} \indent \text{for unit vectors } \mathbf{u} \in \text{Span}\{B_{\cap_{i \in J}\pi(S_i)}\}.
\end{align}
We therefore have by Lemma \ref{MinDimLemma} (since $\delta/\rho < 1$) that $\dim(\text{Span}\{B_{\cap_{i \in J}\pi(S_i)}\}) \leq \dim(\text{Span}\{A_{\cap_{i \in J} S_i}\})$ and \eqref{fact2} follows by the linear independence of the columns of $A_{S_i}$ and $B_{\pi(S_i)}$ for all $i \in [m]$.

Suppose now that $J = \{i-k+1, \ldots, i\}$ so that $\cap_{i \in J} S_i = i$. By \eqref{fact2} we have that $\cap_{i \in J} \pi(S_i)$ is either empty or it contains a single element. Lemma \ref{NonEmptyLemma} ensures that the latter case is the only possibility. Thus, the association $i \mapsto \cap_{i \in J} \pi(S_i)$ defines a map $\hat \pi: [m] \to [m]$. Recalling \eqref{SubspaceMetricSameDim}, it follows from \eqref{fact1} that for all unit vectors $\mathbf{u} \in \text{Span}\{A_i\}$ we have $d\left( \mathbf{u}, \text{Span}\{B_{\hat \pi(i)}\}\right) \leq \delta/\rho$ also. Since $i$ is arbitrary, it follows that for every canonical basis vector $\mathbf{e}_i \in \mathbb{R}^m$, letting $c_i = |A\mathbf{e}_i|_2^{-1}$ and $\varepsilon = \delta/\rho$, there exists some $c'_i \in \mathbb{R}$ such that $|c_iA\mathbf{e}_i - c'_iB\mathbf{e}_{\hat \pi(i)}|_2 \leq \varepsilon$ where $\varepsilon < \frac{L_2(A)}{\sqrt{2}} \min_{j \in [m]} c_i$. This is exactly the supposition in \eqref{1D} and the result follows from the subsequent arguments of Section \ref{DUT}. 
\end{proof}

\begin{remark}
The arguments above can easily be modified to prove Lemma \ref{MainLemma2}. Since Lemma \ref{NonEmptyLemma} assumes $m = m'$, we may not invoke it when $m' > m$ to show that $|\cap_{i \in J} \pi(S_i)| = 1$ for $J = \{i-k+1, \ldots, i\}$. Instead, under the additional assumption that $B$ satisfies spark condition \eqref{SparkCondition}, we can swap the roles of $A$ and $B$ in the proof of \eqref{fact1} to show that $\dim(\text{Span}\{B_{\cap_{i \in J}\pi(S_i)}\}) = \dim(\text{Span}\{A_{\cap_{i \in J} S_i}\})$, from which the same fact then follows. The proof is then completed in much the same way as above by defining a map $\pi: [m] \to [m']$ by the association $i \mapsto \cap_{i \in J} \pi(S_i)$, thereby reducing the proof to the $k=1$ case described in Remark \ref{m'geqmk=1}.
\end{remark}

%\begin{remark} In general, there may exist combinations of fewer supports with intersection $\{i\}$, e.g. if $m \geq 2k-1$ then $S_{i - (k-1)} \cap S_i = \{i\}$. For brevity, we have considered a construction that is valid for any $k < m$.
%\end{remark}

%===================================
% 			REFERENCES
%===================================
\bibliographystyle{IEEEtran}
\bibliography{chazthm_ieee}

\end{document}
