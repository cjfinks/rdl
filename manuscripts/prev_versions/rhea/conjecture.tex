 % ----------------------------------------------------------------
% AMS-LaTeX Paper ************************************************
% **** -----------------------------------------------------------
\documentclass[11pt]{amsart}
\usepackage{amssymb,amsmath}
\usepackage{colordvi}
\usepackage[pdftex]{graphicx}
\usepackage{sidecap}

%\usepackage[left=2.5cm,top=2cm,right=2.5cm,nohead,nofoot]{geometry}


\newtheorem{theorem}{Theorem}
\newtheorem{lemma}{Lemma}
\newtheorem{conjecture}{Conjecture}
\newtheorem{problem}{Problem}
\newtheorem{question}{Question}
\newtheorem{proposition}{Proposition}
\newtheorem{definition}{Definition}
\newtheorem{corollary}{Corollary}
\newtheorem{remark}{Remark}
\newtheorem{example}{Example}
\newcommand{\R}{\mathbb{R}}

\title[A Robust ACS Conjecture]{A Robust ACS Conjecture}


\author[C.J. Hillar]{Christopher J. Hillar}
\address{Mathematical Sciences Research Institute, 17 Gauss Way, Berkeley, CA 94720} 
\address{Redwood Center for Theoretical Neuroscience, University of California, Berkeley, CA 94720} 
\email{chillar@msri.org}

\author[D. Rhea]{Darren Rhea}
\address{Department of Mathematics, University of California, Berkeley, CA 94720} 


% The \author macro works with any number of authors. There are two commands
% used to separate the names and addresses of multiple authors: \And and \AND.
%
% Using \And between authors leaves it to \LaTeX{} to determine where to break
% the lines. Using \AND forces a linebreak at that point. So, if \LaTeX{}
% puts 3 of 4 authors names on the first line, and the last on the second
% line, try using \AND instead of \And before the third author name.

\newcommand{\fix}{\marginpar{FIX}}
\newcommand{\new}{\marginpar{NEW}}


\begin{document}

\maketitle

\section{Introduction}

Let $S_{p,k}$ denote the set of all $k$-sparse vectors in $\mathbb R^p$ ($k$-sparse means at most $k$ nonzero components).

\begin{definition}
We say that $A\in  \mathbb R^{m \times p}$  has $(2k,\delta)$-lower-RIP when
\begin{equation}\label{ACShyp1}
\text{for all $a_1,a_2 \in S_{p,k}$, } \|A(a_1 - a_2)\| \geq  \sqrt{1-\delta} \|a_1-a_2\|.
\end{equation}
\end{definition}

\begin{conjecture}\label{robustACSconj}
Suppose $\varepsilon, \delta \in (0,1)$. Suppose $A,B \in \mathbb R^{m \times p}$ where $A$ has has $(2k,\delta)$-lower-RIP.  
Suppose there is a function $f: S_{p,k} \to S_{p,k}$ satisfying 
the almost recovery condition
\begin{equation}\label{ACShyp2}
\text{for all $a\in S_{p,k}$ with $||a|| \leq 1$,} \ \ ||Aa - Bf(a)|| \leq \varepsilon .
\end{equation}
Then there exists a permutation matrix $P \in \mathbb R^{p \times p}$ and a diagonal matrix $D  \in \mathbb R^{p \times p}$ such that 
\begin{equation}\label{ACShyp3}
\text{for all $k$-sparse $a$ with $||a|| \leq 1$,} \ \ ||f(a) - PDa || \leq \frac{ 2}{\sqrt{1-\delta}} \cdot  \varepsilon .
\end{equation}
\end{conjecture}

There are some natural norms relevant for comparing dictionaries.  For instance, here is the definition of what you might call
the $k$-restricted $\frac{\text{Euclid}}{\text{Euclid}}$ norm: 
\begin{definition}
The set of sparse $a\in S_{p,k}\subseteq \mathbb{R}^p$ such that $\|a\|_\text{Euclid}=1$ is compact.
Therefore for matrices $\underset{m\times p}{M}$we can define the $k$-restricted-Euclidean-over-Euclidean sorta-matrix norm via
\begin{equation}
\|M\|_\text{restricted}:=\underset{\substack{a\in S_{p,k} \\ \|a\|_\text{Euclid}= 1}}{\max}  \|Ma\|_\text{Euclid}.
\end{equation}
This is a ``vector" norm because 
\begin{itemize}
\item $\|M\|_\text{restricted}=0$ if and only if $M=0$.
\item $\|cM\|_\text{restricted}=|c|\cdot \|M\|_\text{restricted}$.
\item $\|M+N\|_\text{restricted}\leq \|M\|_\text{restricted}+\|N\|_\text{restricted}$.
\item the claim that $\|MN\|_\text{restricted}\leq \|M\|_\text{restricted} \cdot \|N\|_\text{restricted}$ is not generally true, thus 
sorta-matrix norm.
\end{itemize}
It is not at all clear that the Euclidean norm should be the ``denominator" norm.  Maybe the $L^1$ norm would be more appropriate
for the denominator, although for small $k$ the difference wouldn't be much.  
\end{definition}

\begin{lemma}
Suppose $k\in \mathbb{Z}_{\geq 1}$.  Suppose $\delta \in (0,1)$.  
Suppose that $\underset{m\times p}{A}$ satisfies $(2k,\delta)$-lower-RIP
(Think of this $A$ as the actually correct sparse dictionary.)
Suppose that $\underset{m\times p}{B}$ and $\underset{p\times p}{P}$ is a permutation matrix and $\underset{p\times p}{D}$
is a diagonal matrix and and $\underset{p\times p}{J}$
is a diagonal matrix whose diagonal elements are $\pm 1$ such that 
 \begin{equation}
 \|J-D\|_\text{spectral}<\varepsilon
 \end{equation}
 and 
\begin{equation}\|D^{-1}\|_\text{spect}<\frac{1}{1-\varepsilon}.
\end{equation}
Suppose  $\underset{m\times p}{C}:=BPD$ has  $\|A-C\|_\text{restricted} \leq \varepsilon$ under the Euclidean over Euclidean $k$-restricted
sorta-norm.   (Think $B$ is estimated dictionary under noise.)   
Suppose that vector $y=Aa$ with $a\in S_{p,k}$ (Think $y$'s true explanation is coefficients $a$ over dictionary $A$.) 
Suppose that $\|Bb-y\|\leq \eta$ for some $b\in S_{p,k}$.  (It was attempted to express $y$ $k$-sparsely with respect to 
the inferred dictionary $B$, and it was accomplished within $\|Bb-y\|=\|BPDc-y\|=\|Cc-y\|\leq \eta$ for $b,c\in S_{p,k}$, where 
$c=D^{-1}P^Tb$.)
Then $\|J^{-1}P^Tb-a\| \leq \frac{\varepsilon \|b\|}{1-\varepsilon}(\frac{1}{\sqrt{1-\delta}}+1)+\frac{\eta}{\sqrt{1-\delta}}$.
\end{lemma}
\begin{proof}
By substitution $y=Aa$ and $\|Cc-y\|\leq \eta$ yield $\|Cc-Aa\|\leq \eta$.
By the definition of the Euclidean/Euclidean sorta-norm and  $c\in S_{p,k}$ we get
$\|(A-C)c\|\leq \varepsilon \|c\|$.  By the triangle inequality $\|A(c-a)\|\leq \varepsilon \|c\| + \eta$.  By $2k$-RIP and $c-a \in S_{p,2k}$, 
$\|c-a\|\leq \frac{\varepsilon \|c\| + \eta}{\sqrt{1-\delta}}$.

Unfortunately, we never learn $c$, only $b$, so we need to estimate
$\|J^{-1}P^Tb-a\|=\|J^{-1}Dc-a\|=\|c-a+(J^{-1}D-I)c\|\leq \|c-a\|+\|(J^{-1}D-I)c\|\leq \|c-a\|+\|J^{-1}D-I\|\cdot \|c\|\leq \|c-a\|+\varepsilon \cdot \|c\|$ since $\|J^{-1}D-I\|=\|J^{-1}(D-J)\|=\|D-J\|<\varepsilon$.
Therefore $\|J^{-1}P^Tb-a\|\leq \|c-a\|+\varepsilon \cdot \|c\|\leq \frac{\varepsilon \|c\| + \eta}{\sqrt{1-\delta}}+\varepsilon \cdot \|c\|$.

Since $\|D^{-1}\|_\text{spectral}\leq \frac{1}{1-\varepsilon}$, $\|c\|\leq \frac{1}{1-\varepsilon}\|b\|$, so 
$\|J^{-1}P^Tb-a\| \leq \frac{\varepsilon \|b\|}{1-\varepsilon}(\frac{1}{\sqrt{1-\delta}}+1)+\frac{\eta}{\sqrt{1-\delta}}$.
\end{proof}


\begin{theorem}
Suppose that $k=1$.  Suppose $\delta \in (0,1)$.  
Suppose that $A$ satisfies $(2k,\delta)$-lower-RIP. 
Suppose $0<\varepsilon < \sqrt{\frac{1-\delta}{2}}$. 
Suppose $f:S_{p,k}\rightarrow S_{p,k}$.
Suppose the almost recovery condition
\begin{equation}\label{ACShyp2}
\text{for all $k$-sparse $a$ with $||a|| \leq 1$,} \ \|Aa - Bf(a)\|_\text{Euclid} \leq \varepsilon .
\end{equation}
 Then there exist a diagonal matrix $\underset{p\times p}{D}$ and a permutation matrix 
 $\underset{p\times p}{P}$ s.t. 
\begin{equation}\label{newnormclaim}
\|A-BPD\|_\text{restricted}\leq \varepsilon.
\end{equation} 
 If in addition you suppose that real matrices $\underset{m \times p}{A}$ and $\underset{m \times p}{B}$ have Euclidean length $1$ columns, we get that there is a diagonal matrix $\underset{p\times p}{J}$ whose diagonal
entries are $\pm1$ such that 
 \begin{equation} \label{deesinversespectralnorm}
 \|J-D\|_\text{spectral}<\varepsilon
 \end{equation}
 and
 \begin{equation} \label{deesinversespectralnorm}
 \|D^{-1}\|_\text{spectral}<\frac{1}{1-\varepsilon}.
 \end{equation}
\end{theorem}
\begin{proof}
Since $k=1$, $f$ taking in a $1$-sparse vector gives out a $1$-sparse vector, so for each $i=1,2,3,\ldots, p$ we can define $c_i$ and $\pi_i$ s.t.
\[f(e_i)=c_i e_{\pi_i}.\]  Each $c_i\neq 0$ because $c_i=0$ would cause a contradiction because the almost recovery condition 
would tell us that $\|Ae_i\|=\|Ae_i-Bf(e_i)\|\leq \varepsilon$ whereas the RIP condition tells us that $\|Ae_i\|\geq \sqrt{1-\delta}>\varepsilon$. 

We claim that $\pi:[p]\rightarrow [p]$ is injective.  To see this, suppose instead that $\pi(i)=\pi(j)$ for $i\neq j$.
By almost recovery, \[\|Ae_i-Bf(e_i)\|\leq \varepsilon,\] and \[Bf(e_i)=B(c_ie_{\pi_i})=\frac{c_i}{c_j} B(c_je_{\pi_j})=\frac{c_i}{c_j} Bf(e_j)\] 
And thus \[\|Ae_i-\frac{c_i}{c_j} Bf(e_j)\|\leq \varepsilon.\]  Also by almost recovery  
\[\|Ae_j-Bf(e_j)\|\leq \varepsilon\] and thus
\[\|\frac{c_i}{c_j}Ae_j-\frac{c_i}{c_j}Bf(e_j)\|\leq \varepsilon \frac{|c_i|}{|c_j|}\] 
Putting these together by triangle inequality gives
\begin{equation}\label{thingythattrianglegives}
\|Ae_i-\frac{c_i}{c_j}Ae_j\| \leq \varepsilon(1+ \frac{|c_i|}{|c_j|}).
\end{equation}
Meanwhile, the lower-RIP condition on $x=e_i- \frac{c_i}{c_j}e_j$ gives
 \[\|A(e_i- \frac{c_i}{c_j}e_j)\|\geq \sqrt{1-\delta} \sqrt{1+\frac{c_i^2}{c_j^2}} > \varepsilon \sqrt{2} \sqrt{1+\frac{c_i^2}{c_j^2}}.\]
 Because $\forall x\in \mathbb{R},\, 1+x\leq \sqrt{2}\sqrt{1+x^2}$, this is a contradiction via $x=\frac{|c_i|}{|c_j|}$ to 
 (\ref{thingythattrianglegives}).  Thus $\pi$ is injective, and thus bijective. 

Let $P$ be the permutation matrix whose $i$-th column is $e_{\pi(i)}$.  Let $D$ be the $p \times p$ diagonal matrix with 
$c_1,c_2, \ldots, c_p$ down the diagonal.  We know that $\|Ae_i-B(c_ie_{\pi(i)})\|\leq \varepsilon$, i.e. that the  $i$-th column $Ae_i$ of $A$ is very close to the $i$-th column of
$BPD$, $BPDe_i=B(c_ie_{\pi(i)})$, and thus
\[\|A-BPD\|_\text{restricted}\leq \varepsilon.\] 

Since the columns $Ae_i$ of $A$ are length one, and the columns $Be_{\pi(i)}$ of $B$ are length one, by the triangle inequality
we get that  $\varepsilon \geq \|Ae_i-B(c_ie_{\pi(i)})\| \geq \left| \|Ae_i\|-|c_i| \|Be_{\pi(i)}\| \right|=|1-|c_i||,$
i.e.  $1-\varepsilon < |c_i| < 1+ \varepsilon$. Thus there is a diagonal matrix $\underset{p\times p}{J}$ whose diagonal
entries are $\pm1$ such that 
 \begin{equation} 
 \|J-D\|_\text{spectral}<\varepsilon
 \end{equation}
 and 
\begin{equation}
\|D^{-1}\|_\text{spect}<\frac{1}{1-\varepsilon}
\end{equation}
 \end{proof}
\section{Ramsey Theory}
Now we think about $k=3$, which will involve the Ramsey theory stuff.
Fix any $1\leq i_1 < i_2 < i_3 \leq p$ for the rest of this argument.
We know that there are subsets $T_1,T_2,T_3 \subseteq [-\frac{1}{\sqrt{3}},\frac{1}{\sqrt{3}}]$ of size 2 and $1\leq r_1 < r_2 < r_3\leq p$ such that  
$\forall t_1 \in T_1, t_2 \in T_2, t_3 \in T_3$ $\exists \alpha_1, \alpha_2, \alpha_3\in \mathbb{R}$ such that 
$\|A(t_1e_{i_1}+t_2 e_{i_2}+t_3e_{i_3})-B(\alpha_1e_{r_1}+\alpha_2 e_{r_2}+\alpha_3 e_{r_3})\|\leq \varepsilon$.

This does not look so good because it is hard to keep the tees in $T_1$ far from each other.

\section{New Thoughts}
Suppose $k=2$.  Suppose  $\underset{4\times 4}{A}$ and $\underset{4\times 4}{B}$ satisfy $(2k,\delta)$-RIP, i.e. they are both
nearly isometries, and they have length one columns.  We define the notion of $A$'s $\{i_1,i_2\}$ ellipse to be the set
\[A\{i_1,i_2\}:=\{A(a_1e_{i_1}+a_2e_{i_2}):\, a_1^2+a_2^2=1\}=\{Aa:\, \|a\|=1, \text{supp}(a)\subseteq \{i_1,i_2\}\}.\]

Similarly we define the notion of $B$'s $\{r_1,r_2\}$ ellipse to be the set
\[B\{r_1,r_2\}:=\{B(b_1e_{r_1}+b_2e_{r_2}):\, b_1^2+b_2^2=1\}=\{Bb:\, \|b\|=1, \text{supp}(b)\subseteq \{r_1,r_2\}\}.\]

For any ellipse $A\{i_1,i_2\}$ there must be a corresponding $B\{r_1,r_2\}$ that minimizes the max distance 
\begin{equation}\underset{p\in A\{i_1,i_2\}}{\max} d(p, B\{r_1,r_2\})\end{equation}
where the Euclidean distance from a point to a compact set is defined in the usual way.
In this manner, the category of signals that $A$ explains as being $\{i_1,i_2\}$-sparse is best explained by 
$B$ as $\{r_1,r_2\}$-sparse, and this suggests a mapping $\tau(\{i_1,i_2\}):=\{r_1,r_2\}$.   We must show that
this map is well-defined, and has certain properties.  
\section{First question} 
If  $B\{r_1,r_2\}$ and $B\{j_1,j_2\}$ are closer than $\sqrt{1-\delta}$  to
each other, does that imply that $\{r_1,r_2\}=\{j_1,j_2\}$?  Yes.
\begin{proof}
Suppose instead that $\{r_1,r_2\}\neq \{j_1,j_2\}$.
WLOG we can assume that $r_1\notin \{j_1,j_2\}$.  Since they are closer than $\sqrt{1-\delta}$ to each other, there 
must be a point $B(b_1e_{j_1}+b_2e_{j_2})$ from $B\{j_1,j_2\}$ which is closer than $\sqrt{1-\delta}$ to $Be_{r_1}$.
But then $\sqrt{1-\delta}> \|B(b_1e_{j_1}+b_2e_{j_2}-e_{r_1})\|\geq \sqrt{1-\delta} \|b_1e_{j_1}+b_2e_{j_2}-e_{r_1}\| \geq \sqrt{1-\delta}$,
which is a contradiction.  
\end{proof}

\section{Second issue}
Fix $i_1, i_2 \in [p]$. Suppose that there is a disk of $A{i_1,i_2}$ that is not closer than $\sqrt{1-\delta}$ to any disk of $B$.  
Then there is a point which is not close to 2-sparse according to $B$.

\section{Coloring Theorems}
Of course we don't get to pick $\epsilon$ and $\delta$.  Define functions 
\[\beta_\text{lower}:= \frac{\sqrt{1-\delta}-\varepsilon}{\sqrt{1+\delta}} - \eta\] and
\[\beta_\text{upper}:= \frac{\sqrt{1+\delta}+\varepsilon}{\sqrt{1-\delta}} + \eta.\]

and then pick $\eta$ and under constraints
\[\eta \sqrt{2} > 2\varepsilon\sqrt{1+\delta} ,\]
\[\sqrt{1-\delta} \cdot \beta_\text{lower} > \sqrt{1+\delta} \cdot \eta \cdot \sqrt{3}+ \varepsilon \sqrt{3},\] to minimize the quantity
\[\frac{2 \varepsilon + \sqrt{1+\delta} \left(\beta_\text{upper}-\beta_\text{lower} +2\eta\right)}{\sqrt{1-\delta}} .\] 

Fix $i_1, i_2 \in [p]$.  Suppose $a_1, a_2, a_3\in \text{span}\{e_{i_1}, e_{i_2}\}$ with $\forall n\, \|a_n\|=1$.
Then by the assumptions, there certainly exist $j_1, j_2, j_3,k_1, k_2, k_3\in [p]$ and $c_1,c_2,c_3, d_1, d_2, d_3 \in \R$ such that
\[\|Aa_1 - B(c_1e_{j_1}+d_1 e_{k_1})\|\leq \varepsilon \]   
\[\|Aa_2 - B(c_2e_{j_2}+d_2 e_{k_2})\|\leq \varepsilon \]   
\[\|Aa_3 - B(c_3e_{j_3}+d_3 e_{k_3})\|\leq \varepsilon \]   
Suppose that $Aa_1, A a_2$ and $Aa_3$ are interpreted by $B$ to be very close to 1-sparse and 
with different major support indices in the sense that the
 $j_1, j_2, j_3\in [p]$ are distinct with the coefficients $|d_1|, |d_2|, |d_3|$ close to zero, say $\forall n=1,2,3$
\[|d_n| \leq \eta.\] 
We do not ask that $k_1, k_2, k_3\in [p]$ be necessarily distinct from each other.  

It will follow from this assumption that the
coefficients $|c_1|, |c_2|, |c_3|$ must be close to one, more particularly it follows that  $\forall n=1,2,3$
 \[\beta_\text{lower} \leq |c_n| \leq \beta_\text{upper},\]
where 
\[\beta_\text{lower}:= \frac{\sqrt{1-\delta}-\varepsilon}{\sqrt{1+\delta}} - \eta\] and
\[\beta_\text{upper}:= \frac{\sqrt{1+\delta}+\varepsilon}{\sqrt{1-\delta}} + \eta\]
via the following lemma:
\begin{lemma}[Beta Bounds]
Suppose that $\|a\|=1$,
\[\|Aa - B(ce_{j}+d e_{k})\|\leq \varepsilon \]   
and $|d|\leq \eta$.  Then 
 \[\beta_\text{lower} \leq |c| \leq \beta_\text{upper},\]
where 
\[\beta_\text{lower}:= \frac{\sqrt{1-\delta}-\varepsilon}{\sqrt{1+\delta}} - \eta\] and
\[\beta_\text{upper}:= \frac{\sqrt{1+\delta}+\varepsilon}{\sqrt{1-\delta}} + \eta.\]
\end{lemma}
\begin{proof}
\begin{align*}
|c|+|d|=\|ce_j\|+\|de_k\|\geq\|ce_j+de_k\|&\geq \frac{\|B(ce_j+de_k)\|}{\sqrt{1+\delta}}\\
&\geq \frac{\|Aa\|-\varepsilon}{\sqrt{1+\delta}} \\
&\geq \frac{\sqrt{1-\delta}-\varepsilon}{\sqrt{1+\delta}}
\end{align*}
Thus 
\[|c|\geq \frac{\sqrt{1-\delta}-\varepsilon}{\sqrt{1+\delta}} - \eta=:\beta_\text{lower}.\]
On the other side
\begin{align*}
|c|-|d|=\|ce_j\|-\|de_k\|\leq\|ce_j+de_k\|&\leq \frac{\|B(ce_j+de_k)\|}{\sqrt{1-\delta}}\\
&\leq \frac{\|Aa\|+\varepsilon}{\sqrt{1-\delta}} \\
&\leq \frac{\sqrt{1+\delta}+\varepsilon}{\sqrt{1-\delta}}
\end{align*}
Thus 
\[|c|\leq \frac{\sqrt{1+\delta}+\varepsilon}{\sqrt{1-\delta}} + \eta=:\beta_\text{upper}.\]
\end{proof}


We will say that $a_1$ is $B$ 1-sparseish with support index $j_1$.
Similarly will say that $a_2$ is $B$ 1-sparseish with support index $j_2$, and  $a_3$ is $B$ 1-sparseish with support index $j_3$. 
Suppose further that it is true that 
\[\sqrt{1-\delta} \cdot \beta_\text{lower}  > \sqrt{1+\delta} \cdot \eta \cdot \sqrt{3}+ \varepsilon \sqrt{3}.\]
Claim: this is a contradiction, i.e. you cannot have on the same circle $\{i_1,i_2\}$ three $B$ 1-sparseish $a$'s with distinct support indices.
\begin{proof}
To see this, by $a_1, a_2, a_3\in \text{span}\{e_{i_1}, e_{i_2}\}$ we can pick $g_1, g_2, g_3\in \R$ such that 
$g_1a_1+g_2a_2 +g_3a_3=0$ and $g_1^2+g_2^2+g_3^2=1$.  
Thus by triangle inequality
\[\|A(g_1a_1+\ldots+g_3a_3)-B[g_1(c_1e_{j_1}+d_1e_{k_1})+\ldots +g_3(c_3e_{j_3}+d_3e_{k_3})]\| \]
\[\leq \varepsilon (|g_1|+|g_2|+|g_3|)\]
i.e.
\[\|B[g_1(c_1e_{j_1}+d_1e_{k_1})+\ldots +g_3(c_3e_{j_3}+d_3e_{k_3})]\| \leq \varepsilon (|g_1|+|g_2|+|g_3|)\]
and triangle inequality
\[\|B(g_1c_1e_{j_1}+\ldots +g_3c_3e_{j_3})\|  - \|B(g_1d_1e_{k_1}+\ldots +g_3d_3e_{k_3})\|\leq \varepsilon (|g_1|+|g_2|+|g_3|)\]
move to the other side
\[\|B(g_1c_1e_{j_1}+\ldots +g_3c_3e_{j_3})\|  \leq \|B(g_1d_1e_{k_1}+\ldots +g_3d_3e_{k_3})\|+ \varepsilon (|g_1|+|g_2|+|g_3|)\]
$B$ is $(4,\delta)$-RIP so
\[\sqrt{1-\delta}\|g_1c_1e_{j_1}+\ldots +g_3c_3e_{j_3}\|  \leq \sqrt{1+\delta}\|g_1d_1e_{k_1}+\ldots +g_3d_3e_{k_3}\|+ \varepsilon (|g_1|+|g_2|+|g_3|)\]
distinctness of  $j_1, j_2, j_3\in [p]$ gives
\[\sqrt{1-\delta}\sqrt{g_1^2c_1^2+\ldots +g_3^2c_3^2}  \leq \sqrt{1+\delta}\|g_1d_1e_{k_1}+\ldots +g_3d_3e_{k_3}\|+ \varepsilon (|g_1|+|g_2|+|g_3|)\]
triangle inequality
\[\sqrt{1-\delta}\sqrt{g_1^2c_1^2+\ldots +g_3^2c_3^2}  \leq \sqrt{1+\delta}(|g_1d_1|+\ldots +|g_3d_3|)+ \varepsilon (|g_1|+|g_2|+|g_3|)\]
$|c|_\text{min} :=\min\{|c_1|,|c_2|,|c_3|\}\geq \beta_\text{lower}$, $|d|_\text{max} :=\max\{|d_1|,|d_2|,|d_3|\}\leq \eta$ gives
\[\sqrt{1-\delta} \cdot \beta_\text{lower}\cdot \sqrt{g_1^2+\ldots +g_3^2}  \leq \sqrt{1+\delta} \cdot \eta  \cdot (|g_1|+\ldots +|g_3|)+ \varepsilon (|g_1|+|g_2|+|g_3|)\]
$g_1^2+g_2^2+g_3^2=1$ and Cauchy-Schwarz
\[\sqrt{1-\delta} \beta_\text{lower}  \leq \sqrt{1+\delta} \cdot \eta \cdot \sqrt{3}+ \varepsilon \sqrt{3}.\]
This is a contradiction.
Thus this circle $A\{i_1,i_2\}$ could have at most two distinct support indices $j_1$ and $j_2$ 
which have $B$ 1-sparseish vectors, but not three.
\end{proof}

\begin{lemma}[Same 1-Sparseish Support Index Implies Close]
Suppose both $\|a_1\|=1$ and $\|a_2\|=1$  are `` $B$ 1-sparseish with the same support index $j\in [p]$" in that 
\[\|Aa_1 - B(c_1e_{j}+d_1 e_{k_1})\|\leq \varepsilon \]   and
\[\|Aa_2 - B(c_2e_{j}+d_2 e_{k_2})\|\leq \varepsilon,\]
where  $|d_1|, |d_2| \leq \eta$.
Suppose also that the sign of $c_1$ is the same as the sign of $c_2$.
Then $a_1$ and $a_2$ must this close to each other:
\[\|a_1-a_2\|\leq   \frac{2 \varepsilon + \sqrt{1+\delta} \left(\beta_\text{upper}-\beta_\text{lower} +2\eta\right)}{\sqrt{1-\delta}}.\]  
\end{lemma}
\begin{proof}
By triangle inequality
\[\|A(a_1-a_2)\|\leq  2\varepsilon + \|B[(c_1-c_2)e_{j}+d_1e_{k_1}-d_2e_{k_2}]\| \] 
by $(4,\delta)$-RIP on $A$ and $B$
\[\sqrt{1-\delta}\|(a_1-a_2)\|\leq  2\varepsilon + \sqrt{1+\delta} \|(c_1-c_2)e_{j}+d_1e_{k_1}-d_2e_{k_2}\| \] 
More triangle inequality
\[\sqrt{1-\delta}\|(a_1-a_2)\|\leq  2\varepsilon + \sqrt{1+\delta} \left(|c_1-c_2|+|d_1|+ |d_2|\right) \] 
$c_1$ and $c_2$ have the same sign and the beta bounds give
\[\sqrt{1-\delta}\|(a_1-a_2)\|\leq  2\varepsilon + \sqrt{1+\delta} \left(\beta_\text{upper}-\beta_\text{lower}+\eta+ \eta \right) \] 
\[\|a_1-a_2\|\leq   \frac{2 \varepsilon + \sqrt{1+\delta} \left(\beta_\text{upper}-\beta_\text{lower} +2\eta\right)}{\sqrt{1-\delta}} \] 
\end{proof}
Thus a given circle $A\{i_1,i_2\}$ can have at most four small regions on it where the $B$ explanation is 1-sparseish, namely
for at most two support indices $j_1$ and $j_2$, and signs on them: positive coefficient times $e_{j_1}$-ish,
positive coefficient times $e_{j_2}$-ish, negative coefficient times $e_{j_1}$-ish, and negative coefficient times $e_{j_2}$-ish.
This is not enough to cover the whole circle $\{i_1,i_2\}$ (imagine a circle with four small sections missing),
so there must a long segment of points on the circle where the $B$ explanation is NOT 1-sparseish, i.e. $B$ explains those points
as a large coefficiented linear combination of some $e_{j_1}$, $e_{j_2}$.  But then either this segment has a 
single color=support set =$\{j_1,j_2\}$ that $B$-explains them all, or there is a point $a$ that has two different color explanations
$\{j_1,j_2\}$ and $\{k_1,k_2\}$.
But the second possibility is a contradiction since the coefficients on this long segment must be large since every point on it is
NOT B 1-sparseish:
\[\|Aa - B(c_1e_{j_1}+c_2 e_{j_2})\|\leq \varepsilon \]
with $|c_1|,|c_2|>\eta$ both big and 
\[\|Aa - B(d_1e_{k_1}+d_2 e_{k_2})\|\leq \varepsilon \]
with $|d_1|,|d_2|>\eta$ both big leads to
 \[\| B(c_1e_{j_1}+c_2 e_{j_2}) - B(d_1e_{k_1}+d_2 e_{k_2})\|\leq 2\varepsilon \]
which by $(4,\delta)$-RIP leads to
\[\| c_1e_{j_1}+c_2 e_{j_2} - d_1e_{k_1}-d_2 e_{k_2}\|\leq 2\varepsilon\sqrt{1+\delta} \]
and at least two of $e_{k_1},e_{k_2},e_{j_1},e_{j_2}$ have no one to cancel with, so 
\[\eta \sqrt{2}\leq 2\varepsilon\sqrt{1+\delta} \]
By the way that $\eta$ was chosen, this is a contradiction.
Thus there is a long almost quarter segment of any circle $\{i_1,i_2\}$ which is ``monochromatic" in its $B$-explanation.   

Let $m_1$ and $m_2$ be two same $\{j_1,j_2\}$-$B$-colored unit vectors on the $\{i_1,i_2\}$ circle at maximally uncorrelated 
angle $\theta$, 
hopefully almost $\approx \pi/2$  radians apart. 
and form a matrix $M=[m1|m2]$.  Let $\|a\|=1$ be any vector on the $\{i_1,i_2\}$ circle.  Certainly there are coefficients $c_1, c_2$
such that $a=c_1 m_1 +c_2 m_2 =Mc$.  Since 
\[\|Am_1-B(d_1 e_{j_1}+d_2e_{j_2})\|\leq \varepsilon \] and 
\[\|Am_2-B(g_1 e_{j_1}+g_2e_{j_2})\|\leq \varepsilon \]
by linear combination and triangle inequality
\[\|A(c_1m_1+c_2m_2)-B((c_1d_1+c_2g_1 )e_{j_1}+(c_1d_2+c_2g_2)e_{j_2})\|\leq (|c_1|+|c_2|)\varepsilon \]
\[\leq \sqrt{2}\cdot \sqrt{c_1^2+c_2^2} \cdot \varepsilon \]
\[\leq \frac{\sqrt{2} \cdot \varepsilon}{1-|\cos(\theta)|} \]
because 
\[\underset{\|c_1 m_1 +c_2 m_2\|=1}{\max} \left(c_1^2+c_2^2\right)=\underset{\|Mc\|=1}{\max} \|c\|^2=\underset{c^TM^TMc=1}{\max} \|c\|^2\]
\[=\frac{1}{\underset{ \|c\|=1}{\min} c^TM^TMc}=\frac{1}{(1-|\cos(\theta)|)^2}\]
since \[M^TM=\left[\begin{array}{cc}1 & \langle m1, m2 \rangle \\ \langle m1 , m2 \rangle & 1 \end{array}\right]
=\left[\begin{array}{cc}1 & \cos(\theta) \\ \cos(\theta) & 1 \end{array}\right]\]
Whose eigenvalues are $1\pm |\cos(\theta)|=1\pm |\langle m_1, m_2 \rangle|$.

Thus we see that the entire $\{i_1,i_2\}$ circle can be $B$-explained as being  $\{j_1,j_2\}$-$B$-colored if you increase the 
tolerance to $\frac{\sqrt{2} \cdot \varepsilon}{1-|\cos(\theta)|}$.
\section{Wedge Products}
Consider $\wedge^2 \R^n=\wedge^2 (\R^n)$, i.e. the span of all the symbols $\{e_i\wedge e_j\,|\, i,j\in [n]\}$ modded out by the 
usual truths like $x_i\wedge y_j=-y_j\wedge x_i$, left and right distributive, constant pullout, etc., familiar from differential forms.

The inner product on $\wedge^2 \R^p$ is defined by 
\begin{equation}
\langle u\wedge v, x \wedge y \rangle:=\det \left[ \begin{array}{c c} 
 \langle u,x \rangle &  \langle u,y \rangle \\ 
 \langle v,x \rangle & \langle v,y \rangle \\ 
 \end{array} \right]
 = \langle u,x \rangle  \langle v,y \rangle -   \langle u,y \rangle  \langle v,x \rangle.
\end{equation}
so in particular
\begin{equation}
\|u\wedge v\|^2=
\langle u\wedge v, u \wedge v \rangle 
= \|u\|^2  \|v\|^2 -   \langle u,v \rangle^2.
\end{equation}
Given $A:\R^p\rightarrow \R^m$, we define $\wedge^2 A: \wedge^2 \R^p \rightarrow \wedge^2 \R^m$ via defining it on the basis 
elements via
\[\wedge^2 A(e_i\wedge e_j) :=(Ae_i) \wedge (Ae_j). \]

Suppose that $A$ satisfies the $(2k, \delta)$-RIP.  We claim that $\wedge^2 A$ satisfies something like it as well.  
First, a lemma:
\begin{lemma}
Suppose that $A$ satisfies the $(2k, \delta)$-RIP.  Then for any $u,v\in S_{p,k}$  
\begin{equation}
\langle u, v \rangle-\frac{1}{2} \delta (\|u\|^2 + \|v\|^2) \leq \langle Au,Av \rangle\leq \langle u, v \rangle+\frac{1}{2} \delta (\|u\|^2 + \|v\|^2).
\end{equation}
\begin{proof}
$u+v, u-v \in S_{p,2k}$ so 
\begin{align*}
4(1-\delta) \langle u, v \rangle-2 \delta (\|u-v\|^2)&=4 \langle u, v \rangle-2 \delta (\|u\|^2 + \|v\|^2) = \\
(1-\delta) \|u+v\|^2- (1+\delta) \|u-v\|^2 \ &\leq \|A(u+v)\|^2- \|A(u-v)\|^2 =4 \langle Au, Av\rangle  \\
&\leq  (1+\delta) \|u+v\|^2- (1-\delta) \|u-v\|^2 \\
&=4 \langle u, v \rangle+2 \delta (\|u\|^2 + \|v\|^2)=\\
&=4(1+\delta) \langle u, v \rangle+2 \delta (\|u-v\|^2)\\ 
\end{align*}
Now divide through by 4.
\end{proof}
\end{lemma}
\bibliographystyle{plain}
\bibliography{acs}


\end{document}
